\documentclass[11pt,oneside]{book}

%%%%%%%%%%%%%%Include Packages%%%%%%%%%%%%%%%%%%%%%%%%%%
\usepackage{xcolor}
\usepackage{colortbl}
\usepackage{tikz}
\usepackage{mathtools}
\usepackage[legalpaper, margin=0.8in]{geometry}
\usepackage{amsmath}
\usepackage{amsthm}
\usepackage{amssymb}
\usepackage{rsfso}
\usepackage{wasysym}
\usepackage{hyperref}
\usetikzlibrary{matrix, calc, arrows,
                arrows.meta, fit,
                positioning, quotes,
                shapes.geometric}


%%%%%%%%%%%%%%%Color%%%%%%%%%%%%%%%%%%%%%%%%%%
\definecolor{gray75}{gray}{0.75}
\definecolor{yellow}{RGB}{255,255,177}
\definecolor{pink}{RGB}{250,204,224}
%%%%%%%%%%%%%%%%%%%%%%%%%%%%%%%%%%%%%%%%%%%%%%%%%%%%%%%%

%%%%%%%%%%%%%%%%%Theorem environments%%%%%%%%%%%%%%%%%%%
\newtheoremstyle{break}
  {\topsep}{\topsep}%
  {\itshape}{}%
  {\bfseries}{}%
  {\newline}{}%
\newtheoremstyle{newStyle}
  {\topsep}{\topsep}%
  {\rmfamily}{}%
  {\bfseries}{}%
  {\newline}{}%             % Theorem head spec
  \theoremstyle{newStyle}
\newtheorem{thm}{Theorem}[chapter]
\newtheorem{lem}{Lemma}[thm]
\newtheorem{axiom}[thm]{Axiom}
\newtheorem{prop}[lem]{Proposition}
\newtheorem{cor}[lem]{Corollary}
\newtheorem{defn}[thm]{Definition}
%%%%%%%%%%%%%%%%%%%%%%%%%%%%%%%%%%%%%%%%%%%%%%%%%%%%%%



\newcommand{\R}{\mathbb{R}}
\newcommand{\N}{\mathbb{N}}
\newcommand{\Z}{\mathbb{Z}}
\newcommand{\Q}{\mathbb{Q}}
\newcommand{\spa}{\text{span}}
\newcommand{\pd}{\partial}
\newcommand{\that}[1]{\widetilde{#1}}
\newcommand{\vmat}[1]{\begin{vmatrix} #1 \end{vmatrix}}
\newcommand{\bmat}[1]{\begin{bmatrix} #1 \end{bmatrix}}
\newcommand{\pmat}[1]{\begin{pmatrix} #1 \end{pmatrix}}
\newcommand*{\Perm}[2]{{}^{#1}\!P_{#2}}
\newcommand*{\Comb}[2]{{}^{#1}C_{#2}}
\newcommand*{\dr}[5]{\draw ({#1}.east) --+ ({#4},{#5}) node[right] ({#2}) {{#3}};}




\newcommand{\note}{\color{red}Note: \color{black}}
\newcommand{\remark}{\color{blue}Remark: \color{black}}
\newcommand{\example}{\color{purple}Example: \color{black}}
\newcommand{\exercise}{\color{cyan}Exercise: \color{black}}





\begin{document}

%%%%%%%%%%%%%%%%%%%%%%%%%%%%%
%% Stat410 Sum24 Lecture 2 %%
%%%%%%%%%%%%%%%%%%%%%%%%%%%%%

\textbf{Example}:
Suppose there is a disease. Let $D$ denote the event that one has the disease. Let $D^c$ denote the event that one does not have the disease. Consider there is a test that gives positive (Pos) and negative (Neg) results. Furthermore, $P(D) = p \in (0,1)$ denotes the probability that a randomly selected person has the disease, and $q \in (0,1)$ denotes the probability that the test identifies the disease correctly. 
\begin{center}
\begin{tikzpicture}[>=triangle 60,every node/.style={anchor=west}]
\matrix (m) [matrix of nodes, row sep=0em, column sep=3em]
{Experimenter\\};
\draw (m-1-1.east) --+ (1,0.7) node[right] (h) {$D$};
\draw (m-1-1.east) --+ (1,-0.7) node[right] (t) {$D^c$};
\draw (h.east) --+ (1.1,1) node[right] (hh) {$ \text{Pos} $};
\draw (h.east) --+ (1.1,0) node[right] (ht) {$ \text{Neg} $};
\draw (t.east) --+ (1,0) node[right] (th) {$ \text{Pos} $};
\draw (t.east) --+ (1,-1) node[right] (tt) {$ \text{Neg} $};
\dr{tt}{TT}{$(D^c,\, \text{Neg} )$}{1}{0}
\dr{th}{TH}{$(D^c,\, \text{Pos} )$}{1}{0}
\dr{ht}{HT}{$(D,\, \text{Neg} )$}{1}{0}
\dr{hh}{HH}{$(D,\, \text{Pos} )$}{1}{0}
\dr{HH}{ohh}{$D$}{1}{0}
\dr{TH}{oth}{$D^c$}{1}{1.4}
\dr{HT}{oht}{$D$}{1}{-1.4}         
\dr{TT}{ott}{$D^c$}{1}{0}
\dr{oth}{oh}{}{1}{0}         
\dr{ohh}{oh}{$ \text{Pos} $}{1.5}{-1}
\dr{ott}{ot}{}{1}{1}
\dr{oht}{ot}{$ \text{Neg} $}{1.35}{0}         
\dr{ot}{obs}{Observer}{1}{0.7}
\dr{oh}{obs}{}{1}{-0.7}

\node[xshift=5.3cm,text width=2.5cm,above=1.9cm] at (m-1-1.north east) {sample space};

\draw (m-1-1.east) edge node[above, rotate=45]{\tiny $p$ \normalsize} (h.west);
\draw (m-1-1.east) edge node[below, rotate=-39]{\tiny $1-p$ \normalsize} (t.west);
\draw (h.east) edge node[above, rotate=50]{\tiny $q$ \normalsize} (hh.west);
\draw (h.east) edge node[below, rotate=0]{\tiny $1-q$ \normalsize} (ht.west);
\draw (t.east) edge node[above, rotate=0]{\tiny $1-q$ \normalsize} (th.west);
\draw (t.east) edge node[below, rotate=-45]{\tiny $q$} (tt.west) ;
\draw (ohh.east) edge node[above, rotate=-35]{\tiny $P(D| \text{Pos} )$} (oh.west) ;
\draw (oth.east) edge node[below, rotate=0]{\tiny $P(D^c| \text{Pos} )$} (oh.west) ;
\draw (oht.east) edge node[above, rotate=0]{\tiny $P(D| \text{Neg} )$} (ot.west) ;
\draw (ott.east) edge node[below, rotate=45]{\tiny $P(D^c| \text{Neg} )$} (ot.west) ;
\draw (ot.east) edge node[below, rotate=45]{\tiny $P( \text{Neg} )$} (obs.west) ;
\draw (oh.east) edge node[above, rotate=-35]{\tiny $P( \text{Pos} )$} (obs.west) ;

\draw[-latex] (0,-2.5) -- ++ (5cm,0) node[midway,below,align=center]{Experiment arrow of time};
\draw[-latex] (15,-2.5) -- ++ (-5cm,0) node[midway,below,align=center]{Observer arrow of time};
\end{tikzpicture}
\end{center}
In this setting, we want $P(\text{Pos}|D^c)$ and $P(\text{Neg}|D)$ to be small, as they correspond to the probabilities of incorrectly identifying the disease using the test. On the other hand, $P(\text{Pos}|D)$ and $P(\text{Neg}|D^c)$ should be large as they correspond to the probabilities of correctly identifying the disease using the test. For the observer, one expects $P(D|\text{Pos})$ and $P(D^c| \text{Neg})$ to be large, and $P(D^c|\text{Pos})$ and $P(D|\text{Neg})$ to be small. 


\vspace{2cm}


\textbf{Example}:
Consider again the disease example. $P(D) = p$, $P(D^c) = 1-p$, and $P(\text{test is correct}) = q$. \\

\qquad\begin{tikzpicture}[>=triangle 60,every node/.style={anchor=west}]
\matrix (m) [matrix of nodes, row sep=0em, column sep=3em]
{Experimenter\\};
\draw (m-1-1.east) --+ (1,0.7) node[right] (h) {$D$};
\draw (m-1-1.east) --+ (1,-0.7) node[right] (t) {$D^c$};
\draw (h.east) --+ (1.1,1) node[right] (hh) {$ \text{Pos} $};
\draw (h.east) --+ (1.1,0) node[right] (ht) {$ \text{Neg} $};
\draw (t.east) --+ (1,0) node[right] (th) {$ \text{Pos} $};
\draw (t.east) --+ (1,-1) node[right] (tt) {$ \text{Neg} $};
\dr{tt}{TT}{$(D^c,\, \text{Neg} )$}{1}{0}
\dr{th}{TH}{$(D^c,\, \text{Pos} )$}{1}{0}
\dr{ht}{HT}{$(D,\, \text{Neg} )$}{1}{0}
\dr{hh}{HH}{$(D,\, \text{Pos} )$}{1}{0}

\node[xshift=5.3cm,text width=2.5cm,above=1.9cm] at (m-1-1.north east) {sample space};

\draw (m-1-1.east) edge node[above, rotate=45]{\tiny $p$ \normalsize} (h.west);
\draw (m-1-1.east) edge node[below, rotate=-39]{\tiny $1-p$ \normalsize} (t.west);
\draw (h.east) edge node[above, rotate=50]{\tiny $q$ \normalsize} (hh.west);
\draw (h.east) edge node[below, rotate=0]{\tiny $1-q$ \normalsize} (ht.west);
\draw (t.east) edge node[above, rotate=0]{\tiny $1-q$ \normalsize} (th.west);
\draw (t.east) edge node[below, rotate=-45]{\tiny $q$} (tt.west) ;
\end{tikzpicture}


Then we can calculate the followings:
\begin{align*}
P(\text{Pos}) &= P(D, \text{Pos})+P(D^c, \text{Pos}) = p \cdot q + (1-p)\cdot (1-q)\,,
\end{align*}
\begin{align*}
P(D|\text{Pos}) &= \frac{P(D, \text{Pos})}{P(\text{Pos})} = \frac{pq}{pq+(1-p)(1-q)}\,, 
\end{align*}
\begin{align*}
P(D^c| \text{Pos}) = \frac{P(D^c, \text{Pos})}{P(\text{Pos})} = \frac{(1-p)(1-q)}{pq+(1-p)(1-q)}\,.
\end{align*}
Similarly, one can calculate
\begin{align*}
P(\text{Neg}) = p\cdot(1-q) + q\cdot (1-p)\,, 
\end{align*}
\begin{align*}
P(D^c|\text{Neg}) = \frac{q(1-p)}{q(1-p)+p(1-q)}\,,\qquad
P(D|\text{Neg}) = \frac{p(1-q)}{q(1-p) +p(1-q)}\,.
\end{align*}
In this setting, we would like to have 
\begin{align*}
P(D|\text{Pos}) = \frac{pq}{pq+(1-p)(1-q)}\,,\qquad
\text{and}\qquad P(D^c|\text{Neg}) = \frac{(1-p)q}{(1-p)q+p(1-q)}
\end{align*}
to be large.\\

Here we consider some special cases: 
\begin{enumerate}
\item Suppose the disease is highly prevalent, for instance, $P(D) = p = 0.9999$. That is, 9999 out of 10000 people on average have the disease. Suppose further that $P(\text{test is correct})= q = 0.9999$. That is, the test is correct 9999 times out of 10000 times on average. For the test to be useful, we want to have both $P(D|\text{Pos})$ and $P(D^c|\text{Neg})$ to be large. In this case, we compute their numerical values
\begin{align*}
P(D|\text{Pos}) = \frac{pq}{pq+(1-p)(1-q)}  = \frac{0.9999 \cdot 0.9999}{0.9999\cdot 0.9999 + 0.0001 \cdot 0.0001}\approx 1\,.
\end{align*}
However, we see that
\begin{align*}
P(D^c|\text{Neg}) = \frac{(1-p)q}{(1-p)q+p(1-q)} = \frac{0.0001 \cdot 0.9999}{0.0001\cdot 0.9999 + 0.9999\cdot 0.0001} = 0.5\,,
\end{align*}
from which we see that a negative test result does not mean the person has no disease. The test is not very useful in this sense. 
\item Now suppose we have a rare disease, for instance, $P(D) = p = 0.0001$. Suppose further that the test accuracy is $P(\text{test is correct}) = 0.9999=q$. For the test to be useful, we again want to have both $P(D|\text{Pos})$ and $P(D^c|\text{Neg})$ to be large, but we again see that 
\begin{align*}
P(D|\text{Pos}) = \frac{pq}{pq+(1-p)(1-q)} = \frac{0.0001\cdot 0.9999}{0.0001 \cdot 0.9999 + 0.9999 \cdot 0.0001} = 0.5\,,
\end{align*}
which is equivalent to a coin toss. 
\end{enumerate}



\vspace{2cm}
\textbf{Example}:
Consider choosing $3$ digits from the set $\{1,2,3,4,5\}$. 
\begin{center}
\begin{tabular}{|c|c|c|}
\hline \rowcolor{lightgray}
 & order matters & order does not matter\\
\hline
\cellcolor{lightgray} with &(1,2,2) is allowed & (1,2,2) is allowed\\
\cellcolor{lightgray} replacement &  but is different from (2,1,2) &  and is the same as (2,1,2)\\
\hline
\cellcolor{lightgray} without &(1,2,2) is not allowed, & (1,2,2) is not allowed,\\
\cellcolor{lightgray} replacement & (1,2,3) is different from (1,3,2) &  (1,2,3) is the same as (1,3,2)\\
\hline
\end{tabular}
\end{center}

\vspace{2cm}
\textbf{Example}:
Consider choosing $2$ digits without replacement from the set $\{1,2,3,4\}$, order does not matter. All the options are $
(1,2),\ (1,3),\ (1,4),\ (2,3),\ (2,4),$ and $(3,4)$. If order matters, more options are available, they are $(2,1)$, $(3,1)$, $(4,1)$, $(3,2)$, $(4,2)$, and $(4,3)$. 

\vspace{2cm}

\textbf{Example}: 
Now consider again choosing $3$ digits from the set of integers $\{1,2,3,4,5\}$. In this case, $n=5$ and $k=3$. We can compute
\begin{align*}
^n\text{P}_k = 
{}^5\text{P}_3 = \frac{5!}{(5-3)!} = 5\cdot 4 \cdot 3 = 60\,, \qquad\quad
^n\text{C}_k = {}^5\text{C}_3
=\frac{5!}{(5-3)!\, 3!} = \frac{60}{3!} = 10\,,
\end{align*}
\begin{align*}
n^k  = 5^3 = 125\,,
\qquad\quad
^{n+k-1}\text{C}_{n-1}=
{}^{7}\text{C}_{4} = \frac{7!}{4!\,3!} = 35\,.
\end{align*}
\begin{center}
\begin{tabular}{|c|c|c|}
\hline \rowcolor{lightgray}
 & order matters & order does not matter\\
\hline
\cellcolor{lightgray} with replacement & 125 & 35\\
\hline
\cellcolor{lightgray} without replacement & 60 & 10\\
\hline
\end{tabular}
\end{center}

\vspace{2cm}
\textbf{Example}:
Given a coin with two sides $H$ and $T$, $P(H) = P(\text{``Coin lands }H") = p \in (0,1)$. In the case where $p = 1/2$, the coin is fair, otherwise not a fair coin. 




\newpage
\begin{tikzpicture}
\tikzset{
    myrectangle/.style={
        draw=black,
        minimum width=8cm,
        minimum height=4cm,
    },
    B/.style={
        draw=blue,
    },
    A/.style={
        draw=orange,
    },
    >=stealth,
    node distance=1cm and 1cm,
}

    \node[myrectangle] (bottom) {};


    % "contents" of bottom node
    \node[xshift=-5cm,text width=2.5cm,above=-1.3cm, rotate=30] at (bottom.north east) {$B\cap A_2$};
    \node[xshift=-6cm,text width=2.5cm,above=-3.5cm, rotate=-20] at (bottom.north east) {$B\cap A_1$};
    \node[xshift=-3.2cm,text width=2.5cm,above=-3.3cm, rotate=10] at (bottom.north east) {$B\cap A_3$};
    \node[xshift=-1.2cm,text width=2.5cm,above=-2.5cm, rotate=30] at (bottom.north east) {$B\cap A_4$};
    \node[xshift=-3cm,text width=2.5cm,above=-1.3cm, rotate=0] at (bottom.north east) {\Large \color{blue!50} $B$};
    \node[xshift=-6.5cm,text width=2.5cm,above=-0.8cm, rotate=0] at (bottom.north east) {\Large \color{orange} $A_1$};
    \node[xshift=-5.3cm,text width=2.5cm,above=-0.8cm, rotate=0] at (bottom.north east) {\Large \color{orange} $A_2$};
    \node[xshift=0.5cm,text width=2.5cm,above=-0.8cm, rotate=0] at (bottom.north east) {\Large \color{orange} $A_4$};\\
    \node[xshift=0.5cm,text width=2.5cm,above=-4.05cm, rotate=0] at (bottom.north east) {\Large \color{orange} $A_5$};
    \node[xshift=-2.25cm,text width=2.5cm,above=-4.05cm, rotate=0] at (bottom.north east) {\Large \color{orange} $A_3$};

    \draw[A] ($(bottom.north west) ! .4 ! (bottom.north east)$) -- ($(bottom.south west) ! .35 ! (bottom.south east)$);
    \draw[A] ($(bottom.north west) ! .6 ! (bottom.north east)$) -- ($(bottom.south west) ! .66 ! (bottom.south east)$);
    \draw[A] ($(bottom.north west) ! .1 ! (bottom.north east)$) -- ($(bottom.south west) ! .35 ! (bottom.south east)$);
    \draw[A] ($(bottom.south west) ! .66 ! (bottom.south east)$) -- ($(bottom.east) ! .3 ! (bottom.south east)$);

    \filldraw[blue!50, opacity = .4] (bottom.center) ellipse [x radius=3.5cm, y radius=1.5cm, rotate=10];
    \node at (bottom.north east) [anchor=south east] {S};

\end{tikzpicture}


\end{document}
