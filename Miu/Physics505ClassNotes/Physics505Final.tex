                                                                                                                                               \documentclass[11pt, onesided]{book}

%%%%%%%%%%%%%%Include Packages%%%%%%%%%%%%%%%%%%%%%%%%%%
\usepackage{xcolor}
\usepackage{mathtools}
\usepackage[a4paper, total={6in, 8in}, margin=1.25in]{geometry}
\usepackage{amsmath}
\usepackage{amssymb}
\usepackage{paralist}
\usepackage{rsfso}
\usepackage{amsthm}
\usepackage{wasysym}
\usepackage[inline]{enumitem}   
\usepackage{hyperref}
\usepackage{tocloft}
\usepackage{wrapfig}
\usepackage{titlesec}
\usepackage{colortbl}
\usepackage{stackengine} 
\usepackage{pdfpages}
%%%%%%%%%%%%%%%%%%%%%%%%%%%%%%%%%%%%%%%%%%%%%%%%%%%%%%%%


%%%%%%%%%%%%%%%Chapter Setting%%%%%%%%%%%%%%%%%%%%%%%%%%
\definecolor{gray75}{gray}{0.75}
\newcommand{\hsp}{\hspace{20pt}}
\titleformat{\chapter}[hang]{\Huge\bfseries}{\thechapter\hsp\textcolor{gray75}{$\mid$}\hsp}{0pt}{\Huge\bfseries}
%%%%%%%%%%%%%%%%%%%%%%%%%%%%%%%%%%%%%%%%%%%%%%%%%%%%%%%%

%%%%%%%%%%%%%%%%%Theorem environments%%%%%%%%%%%%%%%%%%%
\newtheoremstyle{break}
  {\topsep}{\topsep}%
  {\itshape}{}%
  {\bfseries}{}%
  {\newline}{}%
\theoremstyle{break}
\theoremstyle{break}
\newtheorem{axiom}{Axiom}
\newtheorem{thm}{Theorem}[section]
\renewcommand{\thethm}{\arabic{section}.\arabic{thm}}
\newtheorem{lem}{Lemma}[thm]
\newtheorem{cor}{Corollary}[thm]
\newtheorem{defn}{Definition}[thm]
\newenvironment{indEnv}[1][Proof]
  {\proof[#1]\leftskip=1cm\rightskip=1cm}
  {\endproof}
%%%%%%%%%%%%%%%%%%%%%%%%%%%%%%%%%%%%%%%%%%%%%%%%%%%%%%


%%%%%%%%%%%%%%%%%%%%%%%Integral%%%%%%%%%%%%%%%%%%%%%%%
\def\upint{\mathchoice%
    {\mkern13mu\overline{\vphantom{\intop}\mkern7mu}\mkern-20mu}%
    {\mkern7mu\overline{\vphantom{\intop}\mkern7mu}\mkern-14mu}%
    {\mkern7mu\overline{\vphantom{\intop}\mkern7mu}\mkern-14mu}%
    {\mkern7mu\overline{\vphantom{\intop}\mkern7mu}\mkern-14mu}%
  \int}
\def\lowint{\mkern3mu\underline{\vphantom{\intop}\mkern7mu}\mkern-10mu\int}
%%%%%%%%%%%%%%%%%%%%%%%%%%%%%%%%%%%%%%%%%%%%%%%%%%%%%%



\newcommand{\R}{\mathbb{R}}
\newcommand{\N}{\mathbb{N}}
\newcommand{\Z}{\mathbb{Z}}
\newcommand{\Q}{\mathbb{Q}}
\newcommand{\C}{\mathbb{C}}
\newcommand{\T}{\mathcal{T}}
\newcommand{\M}{\mathcal{M}}
\newcommand{\Symm}{\text{Symm}}
\newcommand{\Alt}{\text{Alt}}
\newcommand{\Int}{\text{Int}}
\newcommand{\Bd}{\text{Bd}}
\newcommand{\Power}{\mathcal{P}}
\newcommand{\ee}[1]{\cdot 10^{#1}}
\newcommand{\spa}{\text{span}}
\newcommand{\sgn}{\text{sgn}}
\newcommand{\degr}{\text{deg}}
\newcommand{\pd}{\partial}
\newcommand{\that}[1]{\widetilde{#1}}
\newcommand{\lr}[1]{\left(#1\right)}
\newcommand{\vmat}[1]{\begin{vmatrix} #1 \end{vmatrix}}
\newcommand{\bmat}[1]{\begin{bmatrix} #1 \end{bmatrix}}
\newcommand{\pmat}[1]{\begin{pmatrix} #1 \end{pmatrix}}
\newcommand{\rref}{\xrightarrow{\text{row\ reduce}}}
\newcommand{\txtarrow}[1]{\xrightarrow{\text{#1}}}
\newcommand\oast{\stackMath\mathbin{\stackinset{c}{0ex}{c}{0ex}{\ast}{\Circle}}}
\newcommand{\txt}{Wald's \textit{General Relativity}}

\newcommand{\note}{\color{red}Note: \color{black}}
\newcommand{\remark}{\color{blue}Remark: \color{black}}
\newcommand{\example}{\color{green}Example: \color{black}}
\newcommand{\exercise}{\color{green}Exercise: \color{black}}

%%%%%%%%%%%%%%%%%%%%%%Roman Number%%%%%%%%%%%%%%%%%%%%%%%
\makeatletter
\newcommand*{\rom}[1]{\expandafter\@slowromancap\romannumeral #1@}
\makeatother
%%%%%%%%%%%%%%%%%%%%%%%%%%%%%%%%%%%%%%%%%%%%%%%%%%%%%%%%%

%%%%%%%%%%%%%table of contents%%%%%%%%%%%%%%%%%%%%%%%%%%%%
%\setlength{\cftchapindent}{0em}
%\cftsetindents{section}{2em}{3em}
%
%\renewcommand\cfttoctitlefont{\hfill\huge\bfseries}
%\renewcommand\cftaftertoctitle{\hfill\mbox{}}
%
%\setcounter{tocdepth}{2}
%%%%%%%%%%%%%%%%%%%%%%%%%%%%%%%%%%%%%%%%%%%%%%%%%%%%%%%%%%


%%%%%%%%%%%%%%%%%%%%%Footnotes%%%%%%%%%%%%%%%%%%%%%%%%%%%
\newcommand\blfootnote[1]{%
  \begingroup
  \renewcommand\thefootnote{}\footnote{#1}%
  \addtocounter{footnote}{-1}%
  \endgroup
}
%%%%%%%%%%%%%%%%%%%%%%%%%%%%%%%%%%%%%%%%%%%%%%%%%%%%%%%%%

%%%%%%%%%%%%%%%%%%%%%Section%%%%%%%%%%%%%%%%%%%%%%%%%%%%%
\makeatletter
\def\@seccntformat#1{%
  \expandafter\ifx\csname c@#1\endcsname\c@section\else
  \csname the#1\endcsname\quad
  \fi}
\makeatother
%%%%%%%%%%%%%%%%%%%%%%%%%%%%%%%%%%%%%%%%%%%%%%%%%%%%%%%%%

%%%%%%%%%%%%%%%%%%%%%%%%%%%%%%%%%%%Enumerate%%%%%%%%%%%%%%
\makeatletter
% This command ignores the optional argument 
% for itemize and enumerate lists
\newcommand{\inlineitem}[1][]{%
\ifnum\enit@type=\tw@
    {\descriptionlabel{#1}}
  \hspace{\labelsep}%
\else
  \ifnum\enit@type=\z@
       \refstepcounter{\@listctr}\fi
    \quad\@itemlabel\hspace{\labelsep}%
\fi}
\makeatother
\parindent=0pt
%%%%%%%%%%%%%%%%%%%%%%%%%%%%%%%%%%%%%%%%%%%%%%%%%%%%%%%%%%



\begin{document}

	\begin{titlepage}
		\begin{center}
			\vspace*{0.5cm}
			\Huge \color{red}
				\textbf{Final Exam}\\
			\vspace{0.5cm}			
			\Large \color{black}
			Math 505 - Classical Field Theory\\
			Professor Kai Sun
			\vspace{1.5cm}

			\includegraphics[scale=1.15]{hmm.pdf}
			
			
			\vspace{2cm}
			\LARGE
				\textbf{Jinyan Miao}\\
				\hfill\break
				\LARGE Fall 2023\\
			\vspace{1cm}

		\vspace*{\fill}
		\end{center}			
	\end{titlepage}
 

\newpage
\section{Time Difference between Two Events}
In frame $K$, at time $t_A$, a person $M$ located at $(x_A,y_A,z_A)$ shoots an arrow at a target located at $(x_B, y_B, z_B)$. The arrow hits the target at time $t_B$. An observer in the primed frame $K'$ witnessed the event, and from his perspective, $M$ discharged the arrow at time $t_A'$ with the arrow reaching the target at time $t_B'$.\\

Here we will compute the minimum possible value $t_B'-t_A'$ in terms of space-time coordinates in the $K$ frame. \\

%The spacetime interval is invariant in the two frames, that is,
%\begin{align*}
%c^2(t_B-t_A)^2 - (x_B&-x_A)^2 - (y_B-y_A)^2 - (z_B-z_A)^2 \\
%&= 
%c^2(t'_B-t'_A)^2 - (x'_B-x'_A)^2 - (y'_B-y'_A)^2 - (z'_B-z'_A)^2
%\end{align*}
%As we have timelike events, we we have
%\begin{align*}
%0 \leq c^2(t'_B-t'_A)^2 - (x'_B-x'_A)^2 - (y'_B-y'_A)^2 - (z'_B-z'_A)^2\,.
%\end{align*}
%Thus in order to minimize $t_B' - t_A'$, we would like to 


Assuming that the two frames are moving relative to each other in the $xx'$-direction with speed $v$, from Lorentz transformation, we have
\begin{align*}
t'_B = \gamma(t_B - vx_B/c^2)\,,\qquad
t'_A = \gamma(t_A - vx_A/c^2)\,.
\end{align*}
Thus we can write
\begin{align*}
t'_B - t'_A 
&=\frac{1}{\sqrt{1 - v^2/c^2}} (t_B - t_A) - \frac{v}{c^2}\frac{1}{\sqrt{1 - v^2/c^2}}(x_B-x_A)\\
&=\frac{c}{\sqrt{c^2 - v^2}} (t_B - t_A) - \frac{v/c}{\sqrt{c^2 - v^2}}(x_B-x_A)
\,.
\end{align*}
For simplicity, we denote $t_B - t_A = T$ and $x_B - x_A = X$. Differentiating with respect to $v$ we obtain
\begin{align*}
\frac{d}{dv} (t'_B - t'_A) = \frac{Tv-X}{(c^2 - v^2) \sqrt{1-v^2/c^2}}\,,
\end{align*}
which vanishes when we have
\begin{align*}
Tv-X &= 0\\
v&=X/T = (x_B-x_A) / (t_B-t_A)\,.
\end{align*}
That is, $t_B' - t_A'$ obtains its minimum when $v = (x_B - x_A)/(t_B - t_A)$, in which case we can write
\begin{align*}
t'_B - t'_A &= 
c\left( c^2 -\left( \frac{x_B - x_A}{t_B - t_A}\right)^2
\right)^{-1/2} (t_B - t_A) - \frac{v}{c}\left( c^2 -\left( \frac{x_B - x_A}{t_B - t_A}\right)^2\right)^{-1/2} (x_B - x_A)\\
&= \left(c^2(t_B - t_A)^2 -(x_B - x_A)^2\right)^{1/2}\,c^{-1}\,,
\end{align*}
which is minimized when we have $x_B - x_A$ being maximized. That is, with the assumption that $K'$ and $K$ frame moving with respect to each other in the $xx'$-direction,
if we want to minimize $t_B' - t_A'$, then the arrow should be moving in the $x$-direction in the $K$ frame. In that case, the minimized time interval in $K'$ frame is given by
\begin{align}
t'_B -t_A' = \left(c^2(t_B - t_A)^2 -(x_B - x_A)^2\right)^{1/2} \, c^{-1}\,.
\end{align}
\newpage

This result can be derived via spacetime interval. We can write
\begin{align*}
c^2(t_B-t_A)^2 - (x_B&-x_A)^2 - (y_B-y_A)^2 - (z_B-z_A)^2 \\
&= 
c^2(t'_B-t'_A)^2 - (x'_B-x'_A)^2 - (y'_B-y'_A)^2 - (z'_B-z'_A)^2\,. \tag{2}
\end{align*}
\setcounter{equation}{2}
As we have timelike events, we also have
\begin{align*}
0 \leq c^2(t'_B-t'_A)^2 - (x'_B-x'_A)^2 - (y'_B-y'_A)^2 - (z'_B-z'_A)^2\,.
\end{align*}
The LHS of Eq.\ (2) is fixed, thus $t_B' - t_A'$ is minimized when 
$$(x'_B-x'_A)^2 +(y'_B-y'_A)^2 +(z'_B-z'_A)^2 = 0\,.$$
From Math 513 HW1\footnote{The proof is attached at the end of the text}, we can find a Lorentz frame $K'$ such that the spatial coordinates of event $A$ (arrow is shot by the person) coincide with the spatial coordinates of event $B$ (arrow hits the target). In that case, we simply obtain
\begin{align*}
c^2 ( t'_B - t'_A)^2 = c^2(t_B-t_A)^2 - (x_B&-x_A)^2 - (y_B-y_A)^2 - (z_B-z_A)^2 \,.
\end{align*}
Rearranging we have
\begin{align*}
t'_B - t'_A = \left( 
c^2(t_B-t_A)^2 - (x_B-x_A)^2 - (y_B-y_A)^2 - (z_B-z_A)^2 \right)^{-1/2} c^{-1}\,,
\end{align*}
which agrees with the result that we obtained above, except we have assumed the arrow moves in the $x$-direction when deriving Eq.\ (1).

\newpage
\section{Acceleration of a Charged Particle}
In the context of special relativity, under the influence of electric and magnetic fields, the equation of motion for a particle with a charge $q$ is
\begin{align*}
\frac{d\vec{p}}{dt} = \vec{F} = q\left(\vec{E} + \vec{v}\times \vec{B}\right)\,,
\end{align*}
and the energy obeys
\begin{align*}
\frac{dE}{dt} = \vec{F}\cdot \vec{v} = q\vec{E}\cdot \vec{v}\,.
\end{align*}
We will show that we have
\begin{align*}
\frac{d\vec{v}}{dt} = \frac{q}{m_0}\sqrt{1 - \frac{v^2}{c^2}}\left( \vec{E} + \vec{v}\times \vec{B} - \frac{\vec{v}}{c^2}\left( \vec{v}\cdot \vec{E}\right) \right)\,,
\end{align*}
where $m_0$ is the rest mass of the particle.\\

%Via chain rule, we have
%\begin{align*}
%\frac{d}{dt}(\vec{F}\cdot \vec{v}) = \vec{F}\cdot \dot{\vec{v}} + \dot{\vec{F}}\cdot \vec{v} = q\dot{\vec{E}}\cdot \vec{v} + \dot{\vec{v}}\cdot (q\vec{E})\,.
%\end{align*}
%Rearranging we obtain
%\begin{align*}
%\dot{\vec{v}}\cdot (\vec{v}\times \vec{B}) 
%&=\vec{v}\cdot \left(q\dot{\vec{E}}- \dot{\vec{F}}\right)\\
%&=\vec{v}\cdot \left(q\dot{\vec{E}}- q\dot{\vec{E}} - q\dot{\vec{v}}\times \vec{B} - q\vec{v}\times \dot{\vec{B}} \right)\\
%&=-\vec{v}\cdot \left(  q\dot{\vec{v}}\times \vec{B} + q\vec{v}\times \dot{\vec{B}} \right)\\
%\end{align*}

From class result, we have
\begin{align*}
\frac{d}{dt} (m\vec{v}) = \frac{d}{dt}\left( \frac{m_0\,\vec{v}}{\sqrt{1 - v^2/c^2}}\right) = \frac{d\vec{p}}{dt} = q\left(\vec{E} + \vec{v}\times \vec{B}\right)\,.
\end{align*}
Thus we have
\begin{align*}
\frac{dm}{dt}\, \vec{v} + \frac{d\vec{v}}{dt}m = q(\vec{E}+ \vec{v}\times \vec{B})\,.
\end{align*}
Rearranging we obtain
\begin{align*}
\frac{d\vec{v}}{dt} = 
\frac{q}{m_0}\,\left( 1-\frac{v^2}{c^2}\right)^{1/2}
\left(
(\vec{E}+ \vec{v}\times \vec{B})
-
\frac{\vec{v}}{q} \frac{dm}{dt}\right)\,.
\end{align*}
Thus we are left to evaluate
\begin{align*}
\frac{\vec{v}}{q} \frac{dm}{dt} = \frac{\vec{v}}{qc^2} \frac{dE}{dt} = \frac{\vec{v}}{qc^2}\,(q\vec{E}\cdot \vec{v}) = \frac{\vec{v}}{c^2}\, \vec{E}\cdot \vec{v}\,.
\end{align*}
Combining we obtain
\begin{align*}
\frac{d\vec{v}}{dt} = \frac{q}{m_0}\,\left( 1-\frac{v^2}{c^2}\right)^{1/2}
\left(
(\vec{E}+ \vec{v}\times \vec{B})
-
\frac{\vec{v}}{c^2} (\vec{v}\cdot \vec{E})\right)
\end{align*}
as expected. 

\newpage
\section{Explosion in Special Relativity}
An object with rest mass $m_0$ fragments into two pieces with rest masses $m_A$ and $m_B$ respectively. We will determine the relative velocity between the two fragments.\\

We select an inertial frame where the initial velocity of the object is zero. That is, before the explosion, the object is at rest, thus we have
\begin{align*}
p^\mu = \bmat{m_0 c \\ 0 \\ 0 \\ 0}\,.
\end{align*}
After the explosion, we have
\begin{align*}
p^\mu = \bmat{\gamma_A m_A c \\ 
\gamma_A m_B v_{A,x}\\
\gamma_A m_B v_{A,y}\\
\gamma_A m_B v_{A,z}\\
}+
\bmat{\gamma_B m_B c \\ 
\gamma_B m_B v_{B,x}\\
\gamma_B m_B v_{B,y}\\
\gamma_B m_B v_{B,z}\\
}\,.
\end{align*}
Thus by conservation, we have,
\begin{align*}
\bmat{m_0 c \\ 0 \\ 0 \\ 0}=
\bmat{\gamma_A m_A c \\ 
\gamma_A m_B v_{A,x}\\
\gamma_A m_B v_{A,y}\\
\gamma_A m_B v_{A,z}\\
}+
\bmat{\gamma_B m_B c \\ 
\gamma_B m_B v_{B,x}\\
\gamma_B m_B v_{B,y}\\
\gamma_B m_B v_{B,z}\\
}\,,
\end{align*}
where we have
\begin{align*}
\gamma_B = \frac{1}{\sqrt{1- v_B^2/c^2}}\,,\qquad
\gamma_A = \frac{1}{\sqrt{1- v_A^2/c^2}}\,.
\end{align*}
Thus we have
\begin{align*}
m_0  = \gamma_A m_A + \gamma_B m_B\,,\qquad
m_B \gamma_B \vec{v}_B = -m_A \gamma_A \vec{v}_A\,,
\end{align*}
From which we conclude that the fragments are moving antiparallel, in exactly the opposite direction. Solving the system (according the Physics 505 Class Notes) yields
\begin{align*}
v_A = c\left( 1 - \frac{4m_0^2 m_A^2}{(m_0^2 + m_A^2 - m_B^2)^{2}}\right)^{1/2}\,,\qquad
v_B = c\left( 1 - \frac{4m_0^2 m_B^2}{(m_0^2 + m_B^2 - m_A^2)^{2}}\right)^{1/2}\,.
\end{align*}
Thus in this inertial frame, as the two are moving anti-parallel, the relative speed is
\begin{align*}
v_B - v_A = c\left(\left( 1 - \frac{4m_0^2 m_B^2}{(m_0^2 + m_B^2 - m_A^2)^{2}}\right)^{1/2}- \left( 1 - \frac{4m_0^2 m_A^2}{(m_0^2 + m_A^2 - m_B^2)^{2}}\right)^{1/2}\right)
\end{align*}
We note that in other reference frame, the two fragments are not necessarily moving antiparallel to each other, the computation of the relative velocity between the two in other reference frames would then be more complicated.\\

Here's an outline of the computation of the relative velocity of the two fragments in other reference frames. Suppose in the current frame, the two fragments are moving antiparallel to each other along the $x$-axis, and the initial object exploded at $(0,0,0,0)$. This assumption does not alter any result derived above as translation and rotation of the axes in the current frame does not contradict with our assumption that the object is initially at rest in the current frame. In this case, we can write
\begin{align}
x_A^\mu = (ct,\
v_A t,\
0,\
0)\,,\qquad
x_B^\mu = (ct,\
-v_B t,\
0,\
0)\,,\qquad
\end{align}
Computing the relative velocity of the two fragments in other frames involve Lorentz transforming Eq.\ (3) into other frames and taking the derivative with respect to new coordinate time.\\

\newpage
\section{Energy Flow}
Consider a uniform electric field $\vec{E}$ and a uniform magnetic field $\vec{B}$, the two fields are not parallel to each other. Consider further that there are two observers, one remains stationary in the current frame, and the other moves with a constant velocity $\vec{v}$ perpendicular to both $\vec{E}$ and $\vec{B}$. Both observers attempt to measure the energy current of the field. We will show that they measure the energy current to be in the same direction. \\

According to results from class, the energy current density as measured by the stationary observer has the direction
\begin{align*}
\vec{S} \propto (T^{10}, T^{20}, T^{30}) \propto \vec{E}\times \vec{B}\,,
\end{align*}
with $\vec{E}$ and $\vec{B}$ are measured in the current frame. WLOG, we set up the axes in the current frame such that $\vec{v} = (v,0,0)$. Thus $\vec{B}$ and $\vec{E}$ lie in the $yz$-plane. Now we transform to the primed frame that is moving with velocity $\vec{v}$ with respect to the (unprimed) current frame. Then utilizing result from class, in the primed frame, we have
\begin{align*}
E'_x&= E_x=0 \,,\qquad
E'_y=\gamma(E_y - v B_z)\,,\qquad
E'_z=\gamma(E_z + v B_y)\,,\qquad\\
B'_x&=B_x=0\,,\qquad
B'_y=\gamma(B_y+vE_z/c^2)\,,\qquad
B'_z=\gamma(B_z-vE_y/c^2)\,.
\end{align*}
In this frame, the energy current is thus
\begin{align*}
\vec{S}' \propto \vec{E}'\times \vec{B}'\,.
\end{align*}
We are now left to show $\vec{E}' \times \vec{B}'\propto \vec{E}\times \vec{B}$. From direct computation, we obtain
\begin{align*}
\vec{E}' \times \vec{B}' = 
\gamma^2\bmat{(E_y - vB_z)(B_z - vE_y/c^2)- (E_z + vB_y)(B_y+vE_z/c^2)\\
0\\
0}\propto \bmat{1 \\ 0 \\ 0}\,.
\end{align*}
Notice that on the other hand we have
\begin{align*}
\vec{E}\times \vec{B} \propto  \bmat{1 \\ 0 \\ 0}
\end{align*}
as both $\vec{E}$ and $\vec{B}$ are perpendicular to the $x$-axis. Thus we have shown that we have $\vec{E}\times \vec{B}\propto \vec{E}' \times \vec{B}'$, from which we conclude that $\vec{S}' \propto \vec{S}$, the energy current measured by the two observers have the same direction.\\

\vspace{3cm}
\noindent\rule{8cm}{1pt}\\
\textbf{End of the exam solution}\\
The next two pages display supplement materials to the first problem in the exam.  


\newpage
\includepdf[pages={8,9}]{513HW1.pdf}

\end{document}


