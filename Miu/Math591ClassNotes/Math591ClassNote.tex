\documentclass[11pt]{book}

%%%%%%%%%%%%%%Include Packages%%%%%%%%%%%%%%%%%%%%%%%%%%
\usepackage{xcolor}
\usepackage{mathtools}
\usepackage[a4paper, total={6in, 8in}, margin=1.25in]{geometry}
\usepackage{amsmath}
\usepackage{amssymb}
\usepackage{paralist}
\usepackage{rsfso}
\usepackage{amsthm}
\usepackage{wasysym}
\usepackage[inline]{enumitem}   
\usepackage{hyperref}
\usepackage{tocloft}
\usepackage{wrapfig}
\usepackage{titlesec}
\usepackage{colortbl}
\usepackage{stackengine} 
%%%%%%%%%%%%%%%%%%%%%%%%%%%%%%%%%%%%%%%%%%%%%%%%%%%%%%%%


%%%%%%%%%%%%%%%Chapter Setting%%%%%%%%%%%%%%%%%%%%%%%%%%
\definecolor{gray75}{gray}{0.75}
\newcommand{\hsp}{\hspace{20pt}}
\titleformat{\chapter}[hang]{\Huge\bfseries}{\thechapter\hsp\textcolor{gray75}{$\mid$}\hsp}{0pt}{\Huge\bfseries}
%%%%%%%%%%%%%%%%%%%%%%%%%%%%%%%%%%%%%%%%%%%%%%%%%%%%%%%%

%%%%%%%%%%%%%%%%%Theorem environments%%%%%%%%%%%%%%%%%%%
\newtheoremstyle{break}
  {\topsep}{\topsep}%
  {\itshape}{}%
  {\bfseries}{}%
  {\newline}{}%
\theoremstyle{break}
\theoremstyle{break}
\newtheorem{axiom}{Axiom}
\newtheorem{thm}{Theorem}[section]
\renewcommand{\thethm}{\arabic{section}.\arabic{thm}}
\newtheorem{lem}{Lemma}[thm]
\newtheorem{prop}[lem]{Proposition}
\newtheorem{corL}{Corollary}[lem]
\newtheorem{corT}[lem]{Corollary}
\newtheorem{defn}{Definition}[corL]
\newenvironment{indEnv}[1][Proof]
  {\proof[#1]\leftskip=1cm\rightskip=1cm}
  {\endproof}
%%%%%%%%%%%%%%%%%%%%%%%%%%%%%%%%%%%%%%%%%%%%%%%%%%%%%%


%%%%%%%%%%%%%%%%%%%%%%%Integral%%%%%%%%%%%%%%%%%%%%%%%
\def\upint{\mathchoice%
    {\mkern13mu\overline{\vphantom{\intop}\mkern7mu}\mkern-20mu}%
    {\mkern7mu\overline{\vphantom{\intop}\mkern7mu}\mkern-14mu}%
    {\mkern7mu\overline{\vphantom{\intop}\mkern7mu}\mkern-14mu}%
    {\mkern7mu\overline{\vphantom{\intop}\mkern7mu}\mkern-14mu}%
  \int}
\def\lowint{\mkern3mu\underline{\vphantom{\intop}\mkern7mu}\mkern-10mu\int}
%%%%%%%%%%%%%%%%%%%%%%%%%%%%%%%%%%%%%%%%%%%%%%%%%%%%%%



\newcommand{\R}{\mathbb{R}}
\newcommand{\N}{\mathbb{N}}
\newcommand{\Z}{\mathbb{Z}}
\newcommand{\Q}{\mathbb{Q}}
\newcommand{\C}{\mathbb{C}}
\newcommand{\T}{\mathcal{T}}
\newcommand{\M}{\mathcal{M}}
\newcommand{\Symm}{\text{Symm}}
\newcommand{\Alt}{\text{Alt}}
\newcommand{\Int}{\text{Int}}
\newcommand{\Bd}{\text{Bd}}
\newcommand{\Power}{\mathcal{P}}
\newcommand{\ee}[1]{\cdot 10^{#1}}
\newcommand{\spa}{\text{span}}
\newcommand{\sgn}{\text{sgn}}
\newcommand{\degr}{\text{deg}}
\newcommand{\pd}{\partial}
\newcommand{\that}[1]{\widetilde{#1}}
\newcommand{\lr}[1]{\left(#1\right)}
\newcommand{\vmat}[1]{\begin{vmatrix} #1 \end{vmatrix}}
\newcommand{\bmat}[1]{\begin{bmatrix} #1 \end{bmatrix}}
\newcommand{\pmat}[1]{\begin{pmatrix} #1 \end{pmatrix}}
\newcommand{\rref}{\xrightarrow{\text{row\ reduce}}}
\newcommand{\txtarrow}[1]{\xrightarrow{\text{#1}}}
\newcommand\oast{\stackMath\mathbin{\stackinset{c}{0ex}{c}{0ex}{\ast}{\Circle}}}


\newcommand{\note}{\color{red}Note: \color{black}}
\newcommand{\remark}{\color{blue}Remark: \color{black}}
\newcommand{\example}{\color{green}Example: \color{black}}
\newcommand{\exercise}{\color{green}Exercise: \color{black}}

%%%%%%%%%%%%%%%%%%%%%%Roman Number%%%%%%%%%%%%%%%%%%%%%%%
\makeatletter
\newcommand*{\rom}[1]{\expandafter\@slowromancap\romannumeral #1@}
\makeatother
%%%%%%%%%%%%%%%%%%%%%%%%%%%%%%%%%%%%%%%%%%%%%%%%%%%%%%%%%

%%%%%%%%%%%%table of contents%%%%%%%%%%%%%%%%%%%%%%%%%%%%
\setlength{\cftchapindent}{0em}
\cftsetindents{section}{2em}{3em}

\renewcommand\cfttoctitlefont{\hfill\huge\bfseries}
\renewcommand\cftaftertoctitle{\hfill\mbox{}}

\setcounter{tocdepth}{2}
%%%%%%%%%%%%%%%%%%%%%%%%%%%%%%%%%%%%%%%%%%%%%%%%%%%%%%%%%


%%%%%%%%%%%%%%%%%%%%%Footnotes%%%%%%%%%%%%%%%%%%%%%%%%%%%
\newcommand\blfootnote[1]{%
  \begingroup
  \renewcommand\thefootnote{}\footnote{#1}%
  \addtocounter{footnote}{-1}%
  \endgroup
}
%%%%%%%%%%%%%%%%%%%%%%%%%%%%%%%%%%%%%%%%%%%%%%%%%%%%%%%%%

%%%%%%%%%%%%%%%%%%%%%Section%%%%%%%%%%%%%%%%%%%%%%%%%%%%%
\makeatletter
\def\@seccntformat#1{%
  \expandafter\ifx\csname c@#1\endcsname\c@section\else
  \csname the#1\endcsname\quad
  \fi}
\makeatother
%%%%%%%%%%%%%%%%%%%%%%%%%%%%%%%%%%%%%%%%%%%%%%%%%%%%%%%%%

%%%%%%%%%%%%%%%%%%%%%%%%%%%%%%%%%%%Enumerate%%%%%%%%%%%%%%
\makeatletter
% This command ignores the optional argument 
% for itemize and enumerate lists
\newcommand{\inlineitem}[1][]{%
\ifnum\enit@type=\tw@
    {\descriptionlabel{#1}}
  \hspace{\labelsep}%
\else
  \ifnum\enit@type=\z@
       \refstepcounter{\@listctr}\fi
    \quad\@itemlabel\hspace{\labelsep}%
\fi}
\makeatother
\parindent=0pt
%%%%%%%%%%%%%%%%%%%%%%%%%%%%%%%%%%%%%%%%%%%%%%%%%%%%%%%%%%



\begin{document}

	\begin{titlepage}
		\begin{center}
			\vspace*{1cm}
			\Huge \color{red}
				\textbf{Class Notes}\\
			\vspace{0.5cm}			
			\Large \color{black}
				Math 591 - Differentiable Manifolds\\
				Professor Ralf Spatzier\\	
				University of Michigan\\
			\vspace{2cm}

%%			\includegraphics[scale=0.5]{hmm.pdf}
			
			
			\vspace{4cm}
			\LARGE
				\textbf{Jinyan Miao}\\
				\large \textbf{and his friends from UMich Honors Math}\\
				\hfill\break
				\LARGE Fall 2022\\
			\vspace{1cm}

		\vspace*{\fill}
		\end{center}			
	\end{titlepage}

\newpage 
\tableofcontents
\addtocontents{toc}{~\hfill\textbf{Page}\par}

\newpage
\setcounter{page}{1}
\vspace*{\fill}


\newpage
\chapter{Calculus}

\section[Multivariable Calculus]{\color{red} Multivariable Calculus \color{black}}

\begin{defn}
$S^n$ is a $n-$sphere in $\R^{n+1}$ defined by:
$$S^n \coloneqq \{ x \in \R^{n+1} \mid ||x||^2 = 1\}$$
\end{defn}

To differentiate, we just need a good local structure. 

\begin{defn}
A topological space $M$ is called an $n-$dimensional topological manifold provided that all $p \in M$ has a neighborhood $N$ which is homeomorphic to $\R^n$.
\end{defn}

\begin{defn}
Let $X$ be a collection of subsets of a set $M$. $X$ is said to be locally finite provide that all $p \in M$ has a neighborhood $U$ such that $U$ only intersects finitely many $C \in X$.  
\end{defn}

\begin{defn}
A topological space $M$ is said to be paracompact provided that every open cover $X$ of $M$ admits a finite subcover. 
\end{defn}


\begin{defn}
A cover of a set $M$ is a collection $X$ of sets such that $\bigcup_{C \in X}C = M$
\end{defn}

\begin{defn}
A subcover of $X$ is a cover $X^*$ such that every $U \in X^*$ is contained in some $U \in X$.
\end{defn}

\begin{defn}
A cover $X$ is open provided that all $C \in X$ is open
\end{defn}

\begin{thm}
Topological manifolds are paracompact
\end{thm}

\begin{defn}
$M$ is said to be locally compact provided that all $p \in M$ and neighborhoods $U$ of $p$, there exists a neighborhood $V \subseteq U$ such that $\bar{V}\subseteq U$ is compact. 
\end{defn}

\begin{lem}
Topological manifolds are locally compact
\end{lem}

\begin{defn}
An exhaustion is a sequence of sets $K_n \subseteq K_{n+1}$ such that $\bigcup_n K_n = M$
\end{defn}

\begin{prop}
A secound countable locally compact Hausdorff space admits an exhaustion by compact sets. 
\end{prop}


\end{document}