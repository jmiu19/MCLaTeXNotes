
\documentclass[11pt, oneside]{book}

%%%%%%%%%%%%%%Include Packages%%%%%%%%%%%%%%%%%%%%%%%%%%
\usepackage{xcolor}
\usepackage{mathtools}
\usepackage[a4paper, total={6in, 8in}, margin=1.25in]{geometry}
\usepackage{amsmath}
\usepackage{amssymb}
\usepackage{paralist}
\usepackage{rsfso}
\usepackage{amsthm}
\usepackage{wasysym}
\usepackage[inline]{enumitem}   
\usepackage{hyperref}
\usepackage{tocloft}
\usepackage{wrapfig}
\usepackage{titlesec}
\usepackage{colortbl}
\usepackage{stackengine} 
\usepackage{listings}
%%%%%%%%%%%%%%%%%%%%%%%%%%%%%%%%%%%%%%%%%%%%%%%%%%%%%%%%



%%%%%%%%%%%%%%%Code%%%%%%%%%%%%%%%%%%%%%%%%%%%%%%%%%%%%%
\definecolor{codegreen}{rgb}{0,0.6,0}
\definecolor{codegray}{rgb}{0.5,0.5,0.5}
\definecolor{codepurple}{rgb}{0.58,0,0.82}
\definecolor{backcolour}{rgb}{0.95,0.95,0.92}

\lstdefinestyle{mystyle}{
    backgroundcolor=\color{backcolour},   
    commentstyle=\color{codegreen},
    keywordstyle=\color{magenta},
    numberstyle=\tiny\color{codegray},
    stringstyle=\color{codepurple},
    basicstyle=\ttfamily\footnotesize,
    breakatwhitespace=false,         
    breaklines=true,                 
    captionpos=b,                    
    keepspaces=true,                 
    numbers=left,                    
    numbersep=5pt,                  
    showspaces=false,                
    showstringspaces=false,
    showtabs=false,                  
    tabsize=2
}
%%%%%%%%%%%%%%%%%%%%%%%%%%%%%%%%%%%%%%%%%%%%%%%%%%%%%%%%

%%%%%%%%%%%%%%%Chapter Setting%%%%%%%%%%%%%%%%%%%%%%%%%%
\definecolor{gray75}{gray}{0.75}
\newcommand{\hsp}{\hspace{20pt}}
\titleformat{\chapter}[hang]{\Huge\bfseries}{\thechapter\hsp\textcolor{gray75}{$\mid$}\hsp}{0pt}{\Huge\bfseries}
%%%%%%%%%%%%%%%%%%%%%%%%%%%%%%%%%%%%%%%%%%%%%%%%%%%%%%%%

%%%%%%%%%%%%%%%%%Theorem environments%%%%%%%%%%%%%%%%%%%
\newtheoremstyle{break}
  {\topsep}{\topsep}%
  {\itshape}{}%
  {\bfseries}{}%
  {\newline}{}%
\theoremstyle{break}
\theoremstyle{break}
\newtheorem{axiom}{Axiom}
\newtheorem{thm}{Theorem}[section]
\renewcommand{\thethm}{\arabic{section}.\arabic{thm}}
\newtheorem{lem}{Lemma}[thm]
\newtheorem{cor}{Corollary}[thm]
\newtheorem{defn}{Definition}[thm]
\newenvironment{indEnv}[1][Proof]
  {\proof[#1]\leftskip=1cm\rightskip=1cm}
  {\endproof}
%%%%%%%%%%%%%%%%%%%%%%%%%%%%%%%%%%%%%%%%%%%%%%%%%%%%%%


%%%%%%%%%%%%%%%%%%%%%%%Integral%%%%%%%%%%%%%%%%%%%%%%%
\def\upint{\mathchoice%
    {\mkern13mu\overline{\vphantom{\intop}\mkern7mu}\mkern-20mu}%
    {\mkern7mu\overline{\vphantom{\intop}\mkern7mu}\mkern-14mu}%
    {\mkern7mu\overline{\vphantom{\intop}\mkern7mu}\mkern-14mu}%
    {\mkern7mu\overline{\vphantom{\intop}\mkern7mu}\mkern-14mu}%
  \int}
\def\lowint{\mkern3mu\underline{\vphantom{\intop}\mkern7mu}\mkern-10mu\int}
%%%%%%%%%%%%%%%%%%%%%%%%%%%%%%%%%%%%%%%%%%%%%%%%%%%%%%



\newcommand{\R}{\mathbb{R}}
\newcommand{\N}{\mathbb{N}}
\newcommand{\Z}{\mathbb{Z}}
\newcommand{\Q}{\mathbb{Q}}
\newcommand{\C}{\mathbb{C}}
\newcommand{\T}{\mathcal{T}}
\newcommand{\M}{\mathcal{M}}
\newcommand{\Symm}{\text{Symm}}
\newcommand{\Alt}{\text{Alt}}
\newcommand{\Int}{\text{Int}}
\newcommand{\Bd}{\text{Bd}}
\newcommand{\Power}{\mathcal{P}}
\newcommand{\ee}[1]{\cdot 10^{#1}}
\newcommand{\spa}{\text{span}}
\newcommand{\sgn}{\text{sgn}}
\newcommand{\degr}{\text{deg}}
\newcommand{\pd}{\partial}
\newcommand{\that}[1]{\widetilde{#1}}
\newcommand{\lr}[1]{\left(#1\right)}
\newcommand{\vmat}[1]{\begin{vmatrix} #1 \end{vmatrix}}
\newcommand{\bmat}[1]{\begin{bmatrix} #1 \end{bmatrix}}
\newcommand{\pmat}[1]{\begin{pmatrix} #1 \end{pmatrix}}
\newcommand{\rref}{\xrightarrow{\text{row\ reduce}}}
\newcommand{\txtarrow}[1]{\xrightarrow{\text{#1}}}
\newcommand\oast{\stackMath\mathbin{\stackinset{c}{0ex}{c}{0ex}{\ast}{\Circle}}}
\newcommand{\txt}{Wald's \textit{General Relativity}}

\newcommand{\note}{\color{red}Note: \color{black}}
\newcommand{\remark}{\color{blue}Remark: \color{black}}
\newcommand{\example}{\color{green}Example: \color{black}}
\newcommand{\exercise}{\color{green}Exercise: \color{black}}

%%%%%%%%%%%%%%%%%%%%%%Roman Number%%%%%%%%%%%%%%%%%%%%%%%
\makeatletter
\newcommand*{\rom}[1]{\expandafter\@slowromancap\romannumeral #1@}
\makeatother
%%%%%%%%%%%%%%%%%%%%%%%%%%%%%%%%%%%%%%%%%%%%%%%%%%%%%%%%%

%%%%%%%%%%%%%table of contents%%%%%%%%%%%%%%%%%%%%%%%%%%%%
%\setlength{\cftchapindent}{0em}
%\cftsetindents{section}{2em}{3em}
%
%\renewcommand\cfttoctitlefont{\hfill\huge\bfseries}
%\renewcommand\cftaftertoctitle{\hfill\mbox{}}
%
%\setcounter{tocdepth}{2}
%%%%%%%%%%%%%%%%%%%%%%%%%%%%%%%%%%%%%%%%%%%%%%%%%%%%%%%%%%


%%%%%%%%%%%%%%%%%%%%%Footnotes%%%%%%%%%%%%%%%%%%%%%%%%%%%
\newcommand\blfootnote[1]{%
  \begingroup
  \renewcommand\thefootnote{}\footnote{#1}%
  \addtocounter{footnote}{-1}%
  \endgroup
}
%%%%%%%%%%%%%%%%%%%%%%%%%%%%%%%%%%%%%%%%%%%%%%%%%%%%%%%%%

%%%%%%%%%%%%%%%%%%%%%Section%%%%%%%%%%%%%%%%%%%%%%%%%%%%%
\makeatletter
\def\@seccntformat#1{%
  \expandafter\ifx\csname c@#1\endcsname\c@section\else
  \csname the#1\endcsname\quad
  \fi}
\makeatother
%%%%%%%%%%%%%%%%%%%%%%%%%%%%%%%%%%%%%%%%%%%%%%%%%%%%%%%%%

%%%%%%%%%%%%%%%%%%%%%%%%%%%%%%%%%%%Enumerate%%%%%%%%%%%%%%
\makeatletter
% This command ignores the optional argument 
% for itemize and enumerate lists
\newcommand{\inlineitem}[1][]{%
\ifnum\enit@type=\tw@
    {\descriptionlabel{#1}}
  \hspace{\labelsep}%
\else
  \ifnum\enit@type=\z@
       \refstepcounter{\@listctr}\fi
    \quad\@itemlabel\hspace{\labelsep}%
\fi}
\makeatother
\parindent=0pt
%%%%%%%%%%%%%%%%%%%%%%%%%%%%%%%%%%%%%%%%%%%%%%%%%%%%%%%%%%



\begin{document}

	\begin{titlepage}
		\begin{center}
			\vspace*{0.5cm}
			\Huge \color{red}
				\textbf{Homework 8}\\
			\vspace{0.5cm}			
			\Large \color{black}
			Physics 542 - Quantum Optics\\
			Professor Alex Kuzmich
			\vspace{1.5cm}

			\includegraphics[scale=1.15]{hmm.pdf}
			
			
			\vspace{2cm}
			\LARGE
				\textbf{Jinyan Miao}\\
				\hfill\break
				\LARGE Fall 2023\\
			\vspace{1cm}

		\vspace*{\fill}
		\end{center}			
	\end{titlepage}

\chapter{}
First we compute the commutation relation, for obvious reason we drop the hat ($\hat{•}$) on the operators $\hat{a}$ and $\hat{a}^\dagger$, 
\begin{align*}
[\alpha a^\dagger - a \alpha^* , \beta a^\dagger - a \beta^*]=& 
(\alpha a^\dagger - a \alpha^*)
(\beta a^\dagger - a \beta^*)
-
(\beta a^\dagger - a \beta^*)
(\alpha a^\dagger - a \alpha^*)\\
=& \alpha\beta a^\dagger a^\dagger - \alpha\beta^* a^\dagger a - \alpha^*\beta aa^\dagger + \alpha^*\beta^* aa - \alpha\beta a^\dagger a^\dagger+\alpha^*\beta a^\dagger a+ \alpha\beta^* aa^\dagger -\alpha^*\beta^*aa\\
=& (\alpha^* \beta -\alpha\beta^*)a^\dagger a + (\alpha\beta^* - \alpha^*\beta)aa^\dagger \\
=& (\alpha^* \beta -\alpha\beta^*)a^\dagger a + (\alpha\beta^* - \alpha^*\beta)(1+a^\dagger a) \\
=&\alpha\beta^* - \alpha^*\beta\,.
\end{align*}
Now we can compute
\begin{align*}
D(\alpha,\alpha^*)\, D(\beta, \beta^*) 
&= e^{\alpha a^\dagger  - a\alpha^*} e^{\beta \alpha^\dagger - a\beta^*}\\
&= e^{\alpha a^{\dagger} - a\alpha^{*} + \beta a^\dagger - a\beta^*}e^{[\alpha a^\dagger - a \alpha^* , \beta a^\dagger - a \beta^*]/2}\\
&= e^{(\alpha+\beta)a^\dagger - a(\alpha^*\beta^*)}e^{[\alpha a^\dagger - a \alpha^* , \beta a^\dagger - a \beta^*]/2}\\
&= D(\alpha+\beta, \, \alpha^*+\beta^*) \,e^{(\alpha\beta^* - \alpha^*\beta)/2}\,.
\end{align*}
Thus we can write
\begin{align*}
D(\alpha,\alpha^*)\, D(\beta, \beta^*)\,e^{(\alpha^*\beta-\alpha\beta^* )/2} = D(\alpha+\beta, \, \alpha^*+\beta^*)
\end{align*}
We can consider $e^{(\alpha\beta^* - \alpha^*\beta)/2} = e^{i\Im(\alpha^*\beta)} \in \C$ ($\Im$ is the operator of taking the imaginary part), which has unitary magnitude as the phase of the coherent state. Mathematically, 
$$ D(\alpha, \alpha^*)\, D(\beta, \beta^*)\, |0\rangle \neq D(\alpha+\beta, \alpha^*+\beta^*) |0\rangle = |\alpha+\beta\rangle\,.$$ 
That is $D(\alpha, \alpha^*)\, D(\beta, \beta^*)$ generates a different coherent state than $D(\alpha+\beta, \alpha^*+\beta^*)$. Phase difference between coherent states can be measured, thus the phase $e^{i\Im(\alpha^*\beta)}$ does matter and is not purely a geometrical phase. 

\chapter{}
First we derive the identity
\begin{align*}
e^{\alpha \hat{A}}\hat{B}e^{-\alpha \hat{A}} 
&= 
\left(1+ \alpha\hat{A} + \frac{\alpha^2 \hat{A}^2}{2!} + \frac{\alpha^3 \hat{A}^3}{3!}+\cdots\right)\hat{B}\left(1- \alpha\hat{A} + \frac{\alpha^2 \hat{A}^2}{2!} - \frac{\alpha^3 \hat{A}^3}{3!}+\cdots \right) \\
&= 
\hat{B}+ \alpha(\hat{A}\hat{B} - \hat{B}\hat{A}) + \frac{\alpha^2}{2!}\left( \hat{A}[\hat{A},\hat{B}]-[\hat{A},\hat{B}]\hat{A}\right)+\cdots \\
&= \hat{B} + \alpha [\hat{A},\hat{B}] + \frac{\alpha^2}{2!} [\hat{A},[\hat{A},\hat{B}]]+\cdots\,.
\end{align*}
Now using (6.11) and (6.12), we obtain
\begin{align*}
\hat{a}_2 = \hat{U}^\dagger \hat{a}_0 \hat{U} =e^{-i\pi\hat{J}_1 /2}\hat{a}_0 e^{i\pi \hat{J}_1 /2} = \hat{a}_0 +\left(- i \frac{\pi}{2}\right)[\hat{J}_1, \hat{a}_0] + \frac{1}{2!}\left( \frac{-i \pi}{2}\right)^2 [\hat{J}_1, [\hat{J}_1, \hat{a}_0]]+ \cdots
\end{align*}
where we have abbreviated $\hat{J}_1 =(\hat{a}_0^\dagger \hat{a}_1 + \hat{a}_0 \hat{a}_1^\dagger)/2$. Now we evaluate
\begin{align*}
[\hat{J}_1, \hat{a}_0] &= \left(\hat{a}_0^\dagger \hat{a}_1 + \hat{a}_0 \hat{a}_1^\dagger\right)\hat{a}_0 /2 - \hat{a}_0\left(\hat{a}_0^\dagger \hat{a}_1 + \hat{a}_0 \hat{a}_1^\dagger\right) /2  \\
&=\left( \hat{a}_0^\dagger  \hat{a}_1\hat{a}_0 - \hat{a}_0 \hat{a}_0^\dagger \hat{a}_1 + \hat{a}_0 \hat{a}_1^\dagger \hat{a}_0 - \hat{a}_0 \hat{a}_0 \hat{a}_1^\dagger\right)/2\\
&=\left( \hat{a}_0^\dagger \hat{a}_0 \hat{a}_1 - \hat{a}_0 \hat{a}_0^\dagger \hat{a}_1 + \hat{a}_0\hat{a}_0 \hat{a}_1^\dagger  - \hat{a}_0 \hat{a}_0 \hat{a}_1^\dagger\right)/2\\
&=\left( \hat{a}_0^\dagger \hat{a}_0 \hat{a}_1 - \hat{a}_0 \hat{a}_0^\dagger \hat{a}_1 \right)/2\\
&= \left( [\hat{a}_0^\dagger, \hat{a}_0]\hat{a}_1\right) /2 \\
&= -\hat{a}_1 /2\,.
\end{align*}
Similarly, one is able to obtain
\begin{align*}
[\hat{J}_1,\hat{a}_1] = -\hat{a}_0 /2\,.
\end{align*}
Thus combining we obtain
\begin{align*}
\hat{a}_2 &= \hat{a}_0 +  i\frac{\pi}{4}\hat{a}_1 + \frac{1}{4}\frac{\pi^2}{4}[\hat{J}_1,\hat{a}_1]+ \cdots\\
&= \hat{a}_0 + i\frac{\pi}{4}\hat{a}_1 - \frac{\pi^2}{16}\hat{a}_0- i\frac{\pi^3}{64}\hat{a}_1^3+\cdots\\
&= \cos\left(\frac{\pi}{4}\right) \hat{a}_0 + i\sin\left(\frac{\pi}{4}\right) \hat{a}_1\\
&=\frac{1}{\sqrt{2}}\left( \hat{a}_0 + i\hat{a}_1\right)
\end{align*}
Now we perform similar calculation
\begin{align*}
\hat{a}_3 = \hat{U}^\dagger \hat{a}_1 \hat{U} &=  \hat{a}_1 +\left(- i \frac{\pi}{2}\right)[\hat{J}_1, \hat{a}_1] + \frac{1}{2!}\left( \frac{-i \pi}{2}\right)^2 [\hat{J}_1, [\hat{J}_1, \hat{a}_1]]+ \cdots\\
&= \cos\left( \frac{\pi}{4}\right) \hat{a}_1 + i \sin\left( \frac{\pi}{4}\right) \hat{a}_0 \\
&= \frac{1}{\sqrt{2}}\left( i\hat{a}_ 0 + \hat{a}_1\right)\,.
\end{align*}


\chapter{}
First it is straightforward to write
\begin{align*}
|2\rangle_a |2\rangle_b = \frac{(\hat{a}^\dagger)^2}{\sqrt{
2}}|0\rangle_a
\frac{(\hat{a}^\dagger)^2}{\sqrt{
2}}
 |0\rangle_b = \frac{(\hat{a}^\dagger)^2(\hat{b}^\dagger)^2}{2} |0\rangle_a |0\rangle_b\,.
\end{align*}
We denote the operator for the first and second beam splittersa as $\hat{U}_1$ and $\hat{U}_2$, respectively. Then after the first beam splitter, we should have
\begin{align*}
\hat{U}_1|\text{in}\rangle 
&= \frac{1}{2}\left(\frac{1}{\sqrt{2}}(\hat{a}^\dagger + i\hat{b}^\dagger) \right)^2
\left(\frac{1}{\sqrt{2}}(i\hat{a}^\dagger + \hat{b}^\dagger) \right)^2 |0\rangle_a |0\rangle_b\\
&= \frac{1}{8}(\hat{a}^\dagger+i\hat{b}^\dagger)^2(i\hat{a}^\dagger + \hat{b}^\dagger)^2|0\rangle_a|0\rangle_b\\
&= \frac{1}{8}\left( -(\hat{a}^\dagger)^4 - 2(\hat{a}^\dagger)^2 (\hat{b}^\dagger)^2 -(\hat{b}^\dagger)^4 \right)|0\rangle_a | 0\rangle_b\\
&= -\frac{1}{8}\left( \sqrt{4!}|4\rangle_a | 0\rangle_b + 4|2\rangle_a | 2\rangle_b +\sqrt{4!}|0\rangle_a | 4\rangle_b\right)\,.
\end{align*}
Now we apply the mirror operator $\hat{U}_m$,
\begin{align*}
\hat{U}_m \hat{U}_{1} |\text{in}\rangle = -\frac{1}{8}\left( \sqrt{4!}\left(|4\rangle_a | 0\rangle_b + e^{i4\theta}|0\rangle_a | 4\rangle_b \right)+4e^{i2\theta}|2\rangle_a |2\rangle_b\right)\,.
\end{align*}
In the next calculation we will show that
\begin{align*}
|\text{out}\rangle = 
&+\left(\frac{\sqrt{4!}e^{i2\theta}}{16} - \frac{\sqrt{4!}(1+e^{i4\theta})}{32}\right)\left( |4\rangle_a | 0\rangle_b + |0\rangle_a|4\rangle_b\right)\\
&-\frac{i\sqrt{6}(1-e^{i4\theta})}{8}(|3\rangle_a|1\rangle_b - |1\rangle_a |3\rangle_b)+ \frac{3(1+e^{i4\theta})+ 2e^{i2\theta}}{8}|2\rangle_a|2\rangle_b\,.
\end{align*}
\newpage

We have the second beam splitter, represented by operator $\hat{U}_2$, here we abbreviate $\hat{U}_2 \hat{U}_m \hat{U}_1 = \hat{U}$, then we can write
\begin{align*}
\hat{U}_2\hat{U}_m \hat{U}_1 |\text{in}\rangle 
=& -\frac{1}{8}\left( 
\sqrt{4!}\left( 
\frac{1}{4\sqrt{4!}}\left( \hat{a}^\dagger + i\hat{b}^\dagger\right)^4 |0\rangle_a|0\rangle_b
+\hat{U}e^{i4\theta}|0\rangle_a |4\rangle_b
\right)
+\hat{U} 4e^{i2\theta}|2\rangle_a|2\rangle_b
\right)\\
=& -\frac{1}{8}\left( 
\frac{1}{4}\left(
(\hat{a}^\dagger)^4 + i4(\hat{a}^\dagger)^3 \hat{b}^\dagger - 6(\hat{a}^\dagger)^2(\hat{b}^\dagger)^2 - i4\hat{a}^\dagger(\hat{b}^\dagger)^3 + (\hat{b}^\dagger)^4
\right) |0\rangle_a|0\rangle_b
\right) + \hat{U}\text{(other terms)}\\
=& -\frac{1}{8}\frac{\sqrt{4!}}{4\sqrt{4!}}\left(\sqrt{4!}\left( |4\rangle_a|0\rangle_b + |0\rangle_a|4\rangle_b\right) + i4\sqrt{3!}\left( |3\rangle_a |1\rangle_b - |1\rangle_a|3\rangle_b\right) - 12|2\rangle_a|2\rangle_b\right)\\
&-\frac{e^{i4\theta}}{8}\frac{\sqrt{4!}}{4\sqrt{4!}}\left(\sqrt{4!}\left( |4\rangle_a|0\rangle_b + |0\rangle_a|4\rangle_b\right) - i4\sqrt{3!}\left( |3\rangle_a |1\rangle_b - |1\rangle_a|3\rangle_b\right) - 12|2\rangle_a|2\rangle_b\right)\\
&+ \frac{e^{i2\theta}}{8}\frac{4}{8}\left( \sqrt{4!}\left( |4\rangle_a | 0\rangle_b +|0\rangle_a | 4\rangle_b\right) +4|2\rangle_a |2\rangle_b\right)\\
=& -\frac{\sqrt{4!}(1+e^{i4\theta})}{32}
(|4\rangle_a | 0\rangle_b + |0\rangle_a |4\rangle_b)\\
&-\frac{i4\sqrt{6}(1 - e^{i4\theta})}{32}(|3\rangle_a |1\rangle_b - |1\rangle_a|3\rangle_b) \\
&+ \frac{12(1+e^{i4\theta})}{32}|2\rangle_a |2\rangle_b\\
& +\frac{e^{2i\theta}}{16}\left( \sqrt{4!}\left( |4\rangle_a | 0\rangle_b +|0\rangle_a | 4\rangle_b\right) +4|2\rangle_a |2\rangle_b\right)\\
=&+\left(\frac{\sqrt{4!}e^{i2\theta}}{16} - \frac{\sqrt{4!}(1+e^{i4\theta})}{32}\right)\left( |4\rangle_a | 0\rangle_b + |0\rangle_a|4\rangle_b\right)\tag{first term}\\
&-\frac{i\sqrt{6}(1-e^{i4\theta})}{8}(|3\rangle_a|1\rangle_b - |1\rangle_a |3\rangle_b) \tag{second term}\\
&+ \frac{3(1+e^{i4\theta})+ 2e^{i2\theta}}{8}|2\rangle_a|2\rangle_b\,.\tag{third term}
\end{align*}
It is obvious that the parity operator $\hat{\Pi}_b$ defined by Eq.\,(11.3) does nothing to the first and third terms as the numbers of photons in $b$ states are even, but flips the sign of the second term as number of photons in $b$ state is odd, that is, 
\begin{align*}
\hat{\Pi}_b |\text{out}\rangle  
=&+\left(\frac{\sqrt{4!}e^{i2\theta}}{16} - \frac{\sqrt{4!}(1+e^{i4\theta})}{32}\right)\left( |4\rangle_a | 0\rangle_b + |0\rangle_a|4\rangle_b\right)\\
&+\frac{i\sqrt{6}(1-e^{i4\theta})}{8}(|3\rangle_a|1\rangle_b - |1\rangle_a |3\rangle_b) \\
&+ \frac{3(1+e^{i4\theta})+ 2e^{i2\theta}}{8}|2\rangle_a|2\rangle_b\,.
\end{align*}
Taking the inner product, we obtain
\begin{align*}
\langle \text{out}| \hat{\Pi}_b|\text{out}\rangle = 2\left|\frac{\sqrt{4!}e^{i2\theta}}{16} - \frac{\sqrt{4!}(1+e^{i4\theta})}{32}\right|^2 - 2\left|\frac{i\sqrt{6}(1-e^{i4\theta})}{8}\right|^2 + \left|\frac{3(1+e^{i4\theta})+ 2e^{i2\theta}}{8}\right|^2\,.
\end{align*}
We simplify using Wolframe Mathematica, we obtain
\begin{align*}
\langle \hat{\Pi}_b\rangle =  \langle \text{out}| \hat{\Pi}_b|\text{out}\rangle = \frac{1}{4}\left( 1+ 3\cos(4\theta)\right)\,.
\end{align*}
Lastly, we compute
\begin{align*}
\Delta \theta = \Delta \hat{\Pi}_b \cdot \left|\frac{\pd\langle \hat{\Pi}_b\rangle}{\pd\theta}  \right|^{-1} = \sqrt{1 - \langle \hat{\Pi}_b\rangle^2}\left|\frac{\pd\langle \hat{\Pi}_b\rangle}{\pd\theta}  \right|^{-1}
=\frac{\sqrt{1 - (1+ 3\cos(4\theta))^2/16}}{3|\sin(4\theta)|}\,.
\end{align*}

\chapter{}
After the first beam splitter, we have
\begin{align*}
\hat{U}_1 |\text{in}\rangle = \frac{1}{\sqrt{2}}\left( |N\rangle_a |0\rangle_b + e^{i\phi_N}|0\rangle_a | N\rangle_b \right)\,,
\end{align*}
thus after the mirror, we have
\begin{align*}
|\text{hello} \rangle= \hat{U}_m \hat{U}_1 |\text{in}\rangle = \frac{1}{\sqrt{2}} \left( |N\rangle_a | 0 \rangle_b + e^{i(\phi_N + N\theta)}|0\rangle_a |N\rangle_b\right)
\end{align*}
Following definition from the text,
\begin{align*}
\hat{J}_3 = (\hat{a}^\dagger \hat{a} - \hat{b}^\dagger \hat{b})/2 \,,\qquad \hat{J}_2 = (\hat{a}^\dagger \hat{b} - \hat{b}^\dagger \hat{a})/(2i)\,,\qquad
\hat{J}_0 = (\hat{a}^\dagger \hat{a} + \hat{b}^\dagger \hat{b})/2
\,.
\end{align*}
We see that obviously that $
[\hat{J}_i, \hat{J}_0] = 0$ because, for instance,
\begin{align*}
\hat{J}_1(\hat{a}^\dagger &\hat{a} + \hat{b}^\dagger \hat{b}) - (\hat{a}^\dagger \hat{a} + \hat{b}^\dagger \hat{b})\hat{J}_1 \\
&= \hat{J}_1(\hat{a}^\dagger \hat{a} + \hat{b}^\dagger \hat{b}) - (\hat{a}^\dagger(\hat{J}_1 \hat{a} + \hat{b}) ) + \hat{b}^\dagger(\hat{J}_1 \hat{b}+ \hat{a})\\
&= -\left( -\hat{b}^\dagger \hat{a}+ \hat{a}^\dagger\hat{b} - \hat{a}^\dagger\hat{a} + \hat{b}^\dagger \hat{a}\right) = 0 \,.
\end{align*}
While on the other hand, 
\begin{align*}
[\hat{J}_1,\hat{J}_3 ] &=\frac{1}{2}\left( 
\hat{J}_1
(\hat{a}^\dagger \hat{a} - \hat{b}^\dagger \hat{b})
-
(\hat{a}^\dagger \hat{a} - \hat{b}^\dagger \hat{b})
\hat{J}_1
\right)\\
&= \frac{1}{2}
\left( 
\hat{J}_1
(\hat{a}^\dagger \hat{a} - \hat{b}^\dagger \hat{b})
-
\left(
\hat{a}^\dagger(\hat{J}_1 \hat{a} + \hat{b}/2)-\hat{b}^\dagger (\hat{J}_1\hat{b} + \hat{a}/2)
\right) 
\right)\\
&= -\frac{1}{2}(\hat{a}^\dagger\hat{b} -\hat{b}^\dagger \hat{a} ) = -i \hat{J}_2\,,
\end{align*}
and it can be computed that $[\hat{J}_i, \hat{J}_j] =i\epsilon_{ijk} \hat{J}_k$. Thus $\hat{J}_1,\hat{J}_2,\hat{J}_3$ satisfy those identities of angular momentum operators. We further observe that we have $\hat{J}_0 - \hat{J}_3 = \hat{b}^\dagger \hat{b}$. Also note that $|\text{out}\rangle = e^{i\pi\hat{J}_1 /2} |\text{hello}\rangle$ as shown in Problem 2. Thus it follows that we can write
\begin{align*}
\langle \text{out}| \hat{\Pi}_b |\text{out}\rangle 
&= \langle \text{out}| e^{i\pi (\hat{J}_0 - \hat{J}_3)}|\text{out}\rangle\\
&= \langle \text{hello}|e^{-i \pi\hat{J}_1 /2} e^{i\pi\hat{J}_0 }e^{-i\pi \hat{J}_3} e^{i\pi\hat{J}_1/2} |\text{hello}\rangle\\
&= \langle \text{hello}| e^{i\pi\hat{J}_0 }
e^{-i \pi\hat{J}_1 /2}e^{-i\pi \hat{J}_3} e^{i\pi\hat{J}_1/2} |\text{hello}\rangle\\
&= \langle \text{hello}| e^{i\pi\hat{J}_0 }
e^{-i\pi\hat{J}_2} |\text{hello}\rangle\\
&=\frac{1}{2} \left({}_a\langle N|{}_b\langle 0| + e^{-i(\phi_N + N\theta)}{}_a\langle 0| {}_b\langle N| \right)  e^{i\pi \hat{J}_0}\left(|0\rangle_a|N\rangle_b+ (-1)^Ne^{i(\phi_N +N\theta)}|N\rangle_a | 0\rangle_b\right)\\
&=\frac{1}{2} \left({}_a\langle N|{}_b\langle 0| + e^{-i(\phi_N + N\theta)}{}_a\langle 0| {}_b\langle N| \right)  e^{i\pi N/2}\left(|0\rangle_a|N\rangle_b+ (-1)^Ne^{i(\phi_N +N\theta)}|N\rangle_a | 0\rangle_b\right)\\
&= \frac{e^{i\pi N/2}}{2}\left( (-1)^Ne^{i(\phi_N + N\theta)} + e^{-i(\phi_N + N\theta)}\right)
\end{align*}
which gives $\pm \sin(\phi_N + N\theta) $ for odd $N$ and $\pm\cos(\phi_N + N \theta)$ for even $N$. Now it is not hard to compute, for even $N$, 
\begin{align*}
\Delta \theta = \sqrt{1-\langle \hat{\Pi}_b\rangle}\left|\frac{\pd \langle \hat{\Pi}_b\rangle|}{\pd\phi}\right|^{-1} = \sin(\phi_N + N\theta) \sin^{-1}(\phi_N + N\theta) N^{-1}= \frac{1}{N}
\end{align*}
and for odd $N$, 
\begin{align*}
\Delta \theta = \cos(\phi_N + N \theta) \cos^{-1}(\phi_N +N\theta)N^{-1} = \frac{1}{N}\,,
\end{align*}
as desired. 

\end{document}



