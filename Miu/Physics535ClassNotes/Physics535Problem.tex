\documentclass[11pt, onesided]{book}

%%%%%%%%%%%%%%Include Packages%%%%%%%%%%%%%%%%%%%%%%%%%%
\usepackage{xcolor}
\usepackage{mathtools}
\usepackage[a4paper, total={6in, 8in}, margin=1.25in]{geometry}
\usepackage{amsmath}
\usepackage{amssymb}
\usepackage{paralist}
\usepackage{rsfso}
\usepackage{amsthm}
\usepackage{wasysym}
\usepackage[inline]{enumitem}   
\usepackage{hyperref}
\usepackage{tocloft}
\usepackage{wrapfig}
\usepackage{titlesec}
\usepackage{colortbl}
\usepackage{stackengine} 
%%%%%%%%%%%%%%%%%%%%%%%%%%%%%%%%%%%%%%%%%%%%%%%%%%%%%%%%


%%%%%%%%%%%%%%%Chapter Setting%%%%%%%%%%%%%%%%%%%%%%%%%%
\definecolor{gray75}{gray}{0.75}
\newcommand{\hsp}{\hspace{20pt}}
\titleformat{\chapter}[hang]{\Huge\bfseries}{\thechapter\hsp\textcolor{gray75}{$\mid$}\hsp}{0pt}{\Huge\bfseries}
%%%%%%%%%%%%%%%%%%%%%%%%%%%%%%%%%%%%%%%%%%%%%%%%%%%%%%%%

%%%%%%%%%%%%%%%%%Theorem environments%%%%%%%%%%%%%%%%%%%
\newtheoremstyle{break}
  {\topsep}{\topsep}%
  {\itshape}{}%
  {\bfseries}{}%
  {\newline}{}%
\theoremstyle{break}
\theoremstyle{break}
\newtheorem{axiom}{Axiom}
\newtheorem{thm}{Theorem}[section]
\renewcommand{\thethm}{\arabic{section}.\arabic{thm}}
\newtheorem{lem}{Lemma}[thm]
\newtheorem{cor}{Corollary}[thm]
\newtheorem{defn}{Definition}[thm]
\newenvironment{indEnv}[1][Proof]
  {\proof[#1]\leftskip=1cm\rightskip=1cm}
  {\endproof}
%%%%%%%%%%%%%%%%%%%%%%%%%%%%%%%%%%%%%%%%%%%%%%%%%%%%%%


%%%%%%%%%%%%%%%%%%%%%%%Integral%%%%%%%%%%%%%%%%%%%%%%%
\def\upint{\mathchoice%
    {\mkern13mu\overline{\vphantom{\intop}\mkern7mu}\mkern-20mu}%
    {\mkern7mu\overline{\vphantom{\intop}\mkern7mu}\mkern-14mu}%
    {\mkern7mu\overline{\vphantom{\intop}\mkern7mu}\mkern-14mu}%
    {\mkern7mu\overline{\vphantom{\intop}\mkern7mu}\mkern-14mu}%
  \int}
\def\lowint{\mkern3mu\underline{\vphantom{\intop}\mkern7mu}\mkern-10mu\int}
%%%%%%%%%%%%%%%%%%%%%%%%%%%%%%%%%%%%%%%%%%%%%%%%%%%%%%



\newcommand{\R}{\mathbb{R}}
\newcommand{\N}{\mathbb{N}}
\newcommand{\Z}{\mathbb{Z}}
\newcommand{\Q}{\mathbb{Q}}
\newcommand{\C}{\mathbb{C}}
\newcommand{\T}{\mathcal{T}}
\newcommand{\M}{\mathcal{M}}
\newcommand{\Symm}{\text{Symm}}
\newcommand{\Alt}{\text{Alt}}
\newcommand{\Int}{\text{Int}}
\newcommand{\Bd}{\text{Bd}}
\newcommand{\Power}{\mathcal{P}}
\newcommand{\ee}[1]{\cdot 10^{#1}}
\newcommand{\spa}{\text{span}}
\newcommand{\sgn}{\text{sgn}}
\newcommand{\degr}{\text{deg}}
\newcommand{\pd}{\partial}
\newcommand{\that}[1]{\widetilde{#1}}
\newcommand{\lr}[1]{\left(#1\right)}
\newcommand{\vmat}[1]{\begin{vmatrix} #1 \end{vmatrix}}
\newcommand{\bmat}[1]{\begin{bmatrix} #1 \end{bmatrix}}
\newcommand{\pmat}[1]{\begin{pmatrix} #1 \end{pmatrix}}
\newcommand{\rref}{\xrightarrow{\text{row\ reduce}}}
\newcommand{\txtarrow}[1]{\xrightarrow{\text{#1}}}
\newcommand\oast{\stackMath\mathbin{\stackinset{c}{0ex}{c}{0ex}{\ast}{\Circle}}}
\newcommand{\txt}{Wald's \textit{General Relativity}}

\newcommand{\note}{\color{red}Note: \color{black}}
\newcommand{\remark}{\color{blue}Remark: \color{black}}
\newcommand{\example}{\color{green}Example: \color{black}}
\newcommand{\exercise}{\color{green}Exercise: \color{black}}

%%%%%%%%%%%%%%%%%%%%%%Roman Number%%%%%%%%%%%%%%%%%%%%%%%
\makeatletter
\newcommand*{\rom}[1]{\expandafter\@slowromancap\romannumeral #1@}
\makeatother
%%%%%%%%%%%%%%%%%%%%%%%%%%%%%%%%%%%%%%%%%%%%%%%%%%%%%%%%%

%%%%%%%%%%%%%table of contents%%%%%%%%%%%%%%%%%%%%%%%%%%%%
%\setlength{\cftchapindent}{0em}
%\cftsetindents{section}{2em}{3em}
%
%\renewcommand\cfttoctitlefont{\hfill\huge\bfseries}
%\renewcommand\cftaftertoctitle{\hfill\mbox{}}
%
%\setcounter{tocdepth}{2}
%%%%%%%%%%%%%%%%%%%%%%%%%%%%%%%%%%%%%%%%%%%%%%%%%%%%%%%%%%


%%%%%%%%%%%%%%%%%%%%%Footnotes%%%%%%%%%%%%%%%%%%%%%%%%%%%
\newcommand\blfootnote[1]{%
  \begingroup
  \renewcommand\thefootnote{}\footnote{#1}%
  \addtocounter{footnote}{-1}%
  \endgroup
}
%%%%%%%%%%%%%%%%%%%%%%%%%%%%%%%%%%%%%%%%%%%%%%%%%%%%%%%%%

%%%%%%%%%%%%%%%%%%%%%Section%%%%%%%%%%%%%%%%%%%%%%%%%%%%%
\makeatletter
\def\@seccntformat#1{%
  \expandafter\ifx\csname c@#1\endcsname\c@section\else
  \csname the#1\endcsname\quad
  \fi}
\makeatother
%%%%%%%%%%%%%%%%%%%%%%%%%%%%%%%%%%%%%%%%%%%%%%%%%%%%%%%%%

%%%%%%%%%%%%%%%%%%%%%%%%%%%%%%%%%%%Enumerate%%%%%%%%%%%%%%
\makeatletter
% This command ignores the optional argument 
% for itemize and enumerate lists
\newcommand{\inlineitem}[1][]{%
\ifnum\enit@type=\tw@
    {\descriptionlabel{#1}}
  \hspace{\labelsep}%
\else
  \ifnum\enit@type=\z@
       \refstepcounter{\@listctr}\fi
    \quad\@itemlabel\hspace{\labelsep}%
\fi}
\makeatother
\parindent=0pt
%%%%%%%%%%%%%%%%%%%%%%%%%%%%%%%%%%%%%%%%%%%%%%%%%%%%%%%%%%



\begin{document}

	\begin{titlepage}
		\begin{center}
			\vspace*{0.5cm}
			\Huge \color{red}
				\textbf{Problems}\\
			\vspace{0.5cm}			
			\Large \color{black}
			Physics 535 - General Relativity\\
			Professor Leopoldo A. Pando Zayas
			\vspace{1.5cm}

			\includegraphics[scale=1.15]{hmm.pdf}
			
			
			\vspace{2cm}
			\LARGE
				\textbf{Jinyan Miao}\\
				\hfill\break
				\LARGE Fall 2023\\
			\vspace{1cm}

		\vspace*{\fill}
		\end{center}			
	\end{titlepage}



\tableofcontents
\hfill\break
\hfill\break
\hfill\break
 

\newpage
Consider a metric
\begin{align*}
ds^2 = \frac{dr^2}{1-2\mu/r} + r^2(d\theta^2 + \sin^2(\theta) \, d\phi^2)\,,
\end{align*}
The area of a sphere of radius $R$ is then given by
\begin{align*}
\int \sqrt{\det(g)}\, dx_1\,dx_2\,\cdots dx_n = \int_0^\pi d\theta \, \int_{0}^{2\pi}(R^2 \sin(\theta))\, d\phi = 4\pi R^2\,.
\end{align*}
The radial distance between the sphere $r = 2\mu$ and the sphere $r = 3\mu$ is then given by
\begin{align*}
\int_{2\mu}^{3\mu} \frac{dr}{\sqrt{1- 2\mu/r}} = \left( r\sqrt{1- \frac{2\mu}{r}} + 2\mu \tanh^{-1}\left( \sqrt{1- \frac{2\mu}{r}}\right) \right)|^{r=3\mu}_{r = 2\mu} = (\sqrt{3} + \ln(2+\sqrt{3})) \mu
\end{align*}
The volume of a sphere, characterized by $r > 2\mu$, of radius $r = R$ is given by
\begin{align*}
V = \int_0^\pi \,d\theta \, \int_0^{2\pi}\, d\phi \int_{2\mu}^R \, dr\, \sqrt{\det(g)} = \int_0^\pi \sin(\theta) \, \int_0^{2\pi}\,d\phi \,\int_{2\mu}^R \frac{r^{5/2}}{\sqrt{r- 2\mu}}\,.
\end{align*}

\newpage
The worldline of a particle is described by the parametric equations in some Lorentz frame
\begin{align*}
t(\lambda) = a\sinh(\lambda/a) \,,\qquad
x(\lambda) = a\cosh(\lambda/a)\,,\qquad
y(\lambda) = z(\lambda) = 0\,.
\end{align*}
The particle's four-velocity is given by
\begin{align*}
v^\mu = \frac{dx^\mu}{d\tau} = \left( \cosh(\lambda/a) \, \frac{d\lambda}{d\tau},\ \sinh(\lambda/a)\,\frac{d\lambda}{d\tau},\ 0,\ 0\right)\,.
\end{align*}
To show that $\lambda$ is the proper time along the worldline, we check that we have
\begin{align*}
-1 = g_{\mu\nu}v^\mu v^\nu = \left(
-\cosh^2\left(\frac{\lambda}{a}\right) + \sin^2\left( \frac{\lambda}{a}\right)\right) \,\left( \frac{d\lambda}{d\tau}\right)^2\,,
\end{align*}
from which we see here
\begin{align*}
\frac{d\lambda}{d\tau} = \pm 1\,,
\end{align*}
taking $\lambda = \tau$ we see that the worldline is affinely parametrzied. \\

Here we can find the acceleration of the worldline
\begin{align*}
\alpha^\mu = \frac{dv^\mu}{d\tau} = \left( \frac{1}{a}\sinh\left( \frac{\lambda}{a}\right),\ \frac{1}{a}\cosh\left( \frac{\lambda}{a}\right),\ 0,\ 0\right)\,.
\end{align*}
One thing to notice here is $\alpha^\mu$ is orthogonal to $v^\mu$, as we see here
\begin{align*}
0 = \frac{d}{d\tau}\left( g_{\mu\nu}v^\mu v^\nu\right) = 2g_{\mu\nu}\alpha^\mu v^\nu
\end{align*}
as we have $g_{\mu\nu}v^\mu v^\nu$. Note further here $\alpha^\mu$ is constant 
\begin{align*}
|\alpha^\mu|^2 = \frac{1}{a}\,.
\end{align*}

The usual velocity vector is given by
\begin{align*}
v_x = \frac{dx}{dt} = \frac{dx/d\tau}{dt/d\tau} = \tanh\left( \frac{\tau}{a}\right) = \frac{\sinh(\tau/a)}{\sqrt{\sinh^2(\tau/a) + 1}} = \frac{t/a}{\sqrt{(t/a)^2+1}}
\end{align*}


\newpage
Consider a $2$-space with metric
\begin{align*}
ds^2 = \frac{dr^2 + r^2\, d\theta^2}{r^2 - a^2} - \frac{r^2 \, dr^2}{(r^2 - a^2)^2} =\frac{(dr^2 + r^2\, d\theta^2)(r^2 - a^2) - r^2 \,dr^2}{(r^2 - a^2)^2} =-\frac{a^2\, dr^2}{(r^2 - a^2)^2} + \frac{r^2\, d\theta^2}{r^2 - a^2}\,.
\end{align*}

For null geodesics, $ds^2 =0$, thus we have
\begin{align*}
0 = -\frac{a^2\, dr^2}{(r^2 - a^2)^2} + \frac{r^2\, d\theta^2}{r^2 - a^2}\,,
\end{align*}
or written in parameters
\begin{align*}
-\frac{a^2 \, \dot{r}^2}{(r^2 - a^2)^2} + \frac{r^2 \dot{\theta}^2}{r^2 - a^2} = 0\,.
\end{align*}
Note further that we have
\begin{align*}
\frac{dr}{d\theta} = \frac{\dot{r}}{\theta}\,,
\end{align*}
thus we have
\begin{align*}
a^2\left( \frac{dr}{d\theta}\right)^2 + a^2 r^2 = r^4\,.
\end{align*}
For finding the geodesics, we minimize the integral
\begin{align*}
\int \frac{1}{2}g_{\mu\nu}x^\mu x^\nu \, d\tau\,,
\end{align*}
where we employ Euler-Lagrange equation to minimize
\begin{align*}
\mathcal{L} = \frac{1}{2}\left( \frac{-a^2\dot{r}^2}{(r^2 -a^2)^2} + \frac{r^2 \dot{\theta}^2}{r^2 - a^2}\right)\,,
\end{align*}
we get
\begin{align*}
\frac{d}{dt}\frac{\pd \mathcal{L}}{\pd \dot{\theta}} = \frac{d}{dt}\left(\frac{r^2 \dot{\theta}}{r^2 - a^2}  \right)\coloneqq \frac{d}{dt}\, L = 0\,.
\end{align*}

For timelike geodesic, we can write
\begin{align*}
1 = \frac{a^2\, \dot{r}^2}{(r^2 - a^2)} - \frac{r^2\, \dot{\theta}^2}{r^2 - a^2}
\end{align*}





\end{document}


