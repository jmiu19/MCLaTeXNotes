\documentclass[11pt, onesided]{book}

%%%%%%%%%%%%%%Include Packages%%%%%%%%%%%%%%%%%%%%%%%%%%
\usepackage{xcolor}
\usepackage{mathtools}
\usepackage[a4paper, total={6in, 8in}, margin=1.25in]{geometry}
\usepackage{amsmath}
\usepackage{amssymb}
\usepackage{paralist}
\usepackage{rsfso}
\usepackage{amsthm}
\usepackage{wasysym}
\usepackage[inline]{enumitem}   
\usepackage{hyperref}
\usepackage{tocloft}
\usepackage{wrapfig}
\usepackage{titlesec}
\usepackage{colortbl}
\usepackage{stackengine} 
%%%%%%%%%%%%%%%%%%%%%%%%%%%%%%%%%%%%%%%%%%%%%%%%%%%%%%%%


%%%%%%%%%%%%%%%Chapter Setting%%%%%%%%%%%%%%%%%%%%%%%%%%
\definecolor{gray75}{gray}{0.75}
\newcommand{\hsp}{\hspace{20pt}}
\titleformat{\chapter}[hang]{\Huge\bfseries}{\thechapter\hsp\textcolor{gray75}{$\mid$}\hsp}{0pt}{\Huge\bfseries}
%%%%%%%%%%%%%%%%%%%%%%%%%%%%%%%%%%%%%%%%%%%%%%%%%%%%%%%%

%%%%%%%%%%%%%%%%%Theorem environments%%%%%%%%%%%%%%%%%%%
\newtheoremstyle{break}
  {\topsep}{\topsep}%
  {\itshape}{}%
  {\bfseries}{}%
  {\newline}{}%
\theoremstyle{break}
\theoremstyle{break}
\newtheorem{axiom}{Axiom}
\newtheorem{thm}{Theorem}[section]
\renewcommand{\thethm}{\arabic{section}.\arabic{thm}}
\newtheorem{lem}{Lemma}[thm]
\newtheorem{cor}{Corollary}[thm]
\newtheorem{defn}{Definition}[thm]
\newenvironment{indEnv}[1][Proof]
  {\proof[#1]\leftskip=1cm\rightskip=1cm}
  {\endproof}
%%%%%%%%%%%%%%%%%%%%%%%%%%%%%%%%%%%%%%%%%%%%%%%%%%%%%%


%%%%%%%%%%%%%%%%%%%%%%%Integral%%%%%%%%%%%%%%%%%%%%%%%
\def\upint{\mathchoice%
    {\mkern13mu\overline{\vphantom{\intop}\mkern7mu}\mkern-20mu}%
    {\mkern7mu\overline{\vphantom{\intop}\mkern7mu}\mkern-14mu}%
    {\mkern7mu\overline{\vphantom{\intop}\mkern7mu}\mkern-14mu}%
    {\mkern7mu\overline{\vphantom{\intop}\mkern7mu}\mkern-14mu}%
  \int}
\def\lowint{\mkern3mu\underline{\vphantom{\intop}\mkern7mu}\mkern-10mu\int}
%%%%%%%%%%%%%%%%%%%%%%%%%%%%%%%%%%%%%%%%%%%%%%%%%%%%%%



\newcommand{\R}{\mathbb{R}}
\newcommand{\N}{\mathbb{N}}
\newcommand{\Z}{\mathbb{Z}}
\newcommand{\Q}{\mathbb{Q}}
\newcommand{\C}{\mathbb{C}}
\newcommand{\T}{\mathcal{T}}
\newcommand{\M}{\mathcal{M}}
\newcommand{\Symm}{\text{Symm}}
\newcommand{\Alt}{\text{Alt}}
\newcommand{\Int}{\text{Int}}
\newcommand{\Bd}{\text{Bd}}
\newcommand{\Power}{\mathcal{P}}
\newcommand{\ee}[1]{\cdot 10^{#1}}
\newcommand{\spa}{\text{span}}
\newcommand{\sgn}{\text{sgn}}
\newcommand{\degr}{\text{deg}}
\newcommand{\pd}{\partial}
\newcommand{\that}[1]{\widetilde{#1}}
\newcommand{\lr}[1]{\left(#1\right)}
\newcommand{\vmat}[1]{\begin{vmatrix} #1 \end{vmatrix}}
\newcommand{\bmat}[1]{\begin{bmatrix} #1 \end{bmatrix}}
\newcommand{\pmat}[1]{\begin{pmatrix} #1 \end{pmatrix}}
\newcommand{\rref}{\xrightarrow{\text{row\ reduce}}}
\newcommand{\txtarrow}[1]{\xrightarrow{\text{#1}}}
\newcommand\oast{\stackMath\mathbin{\stackinset{c}{0ex}{c}{0ex}{\ast}{\Circle}}}
\newcommand{\txt}{Wald's \textit{General Relativity}}

\newcommand{\note}{\color{red}Note: \color{black}}
\newcommand{\remark}{\color{blue}Remark: \color{black}}
\newcommand{\example}{\color{green}Example: \color{black}}
\newcommand{\exercise}{\color{green}Exercise: \color{black}}

%%%%%%%%%%%%%%%%%%%%%%Roman Number%%%%%%%%%%%%%%%%%%%%%%%
\makeatletter
\newcommand*{\rom}[1]{\expandafter\@slowromancap\romannumeral #1@}
\makeatother
%%%%%%%%%%%%%%%%%%%%%%%%%%%%%%%%%%%%%%%%%%%%%%%%%%%%%%%%%

%%%%%%%%%%%%%table of contents%%%%%%%%%%%%%%%%%%%%%%%%%%%%
%\setlength{\cftchapindent}{0em}
%\cftsetindents{section}{2em}{3em}
%
%\renewcommand\cfttoctitlefont{\hfill\huge\bfseries}
%\renewcommand\cftaftertoctitle{\hfill\mbox{}}
%
%\setcounter{tocdepth}{2}
%%%%%%%%%%%%%%%%%%%%%%%%%%%%%%%%%%%%%%%%%%%%%%%%%%%%%%%%%%


%%%%%%%%%%%%%%%%%%%%%Footnotes%%%%%%%%%%%%%%%%%%%%%%%%%%%
\newcommand\blfootnote[1]{%
  \begingroup
  \renewcommand\thefootnote{}\footnote{#1}%
  \addtocounter{footnote}{-1}%
  \endgroup
}
%%%%%%%%%%%%%%%%%%%%%%%%%%%%%%%%%%%%%%%%%%%%%%%%%%%%%%%%%

%%%%%%%%%%%%%%%%%%%%%Section%%%%%%%%%%%%%%%%%%%%%%%%%%%%%
\makeatletter
\def\@seccntformat#1{%
  \expandafter\ifx\csname c@#1\endcsname\c@section\else
  \csname the#1\endcsname\quad
  \fi}
\makeatother
%%%%%%%%%%%%%%%%%%%%%%%%%%%%%%%%%%%%%%%%%%%%%%%%%%%%%%%%%

%%%%%%%%%%%%%%%%%%%%%%%%%%%%%%%%%%%Enumerate%%%%%%%%%%%%%%
\makeatletter
% This command ignores the optional argument 
% for itemize and enumerate lists
\newcommand{\inlineitem}[1][]{%
\ifnum\enit@type=\tw@
    {\descriptionlabel{#1}}
  \hspace{\labelsep}%
\else
  \ifnum\enit@type=\z@
       \refstepcounter{\@listctr}\fi
    \quad\@itemlabel\hspace{\labelsep}%
\fi}
\makeatother
\parindent=0pt
%%%%%%%%%%%%%%%%%%%%%%%%%%%%%%%%%%%%%%%%%%%%%%%%%%%%%%%%%%



\begin{document}

	\begin{titlepage}
		\begin{center}
			\vspace*{0.5cm}
			\Huge \color{red}
				\textbf{Practice Problems}\\
			\vspace{0.5cm}			
			\Large \color{black}
			\textit{General Relativity} by Robert Wald\\
			\vspace{1.5cm}

			\includegraphics[scale=1.15]{hmm.pdf}
			
			
			\vspace{2cm}
			\LARGE
				\textbf{Jinyan Miao}\\
				\hfill\break
				\LARGE Winter 2023\\
			\vspace{1cm}

		\vspace*{\fill}
		\end{center}			
	\end{titlepage}


\newpage
\section{Chapter 3 Problem 1}
Consider a derivative operator $\nabla_a$ on manifold $M$, not necessarily torsion free. \\

(a) Show that there exists a tensor $T^c{}_{ab}$ such that for all $f \in C^\infty(M, \R)$, we have $\nabla_a \nabla_b f  - \nabla_b\nabla_a f = -T^c{}_{ab} \nabla_cf$. 
\begin{proof}
For arbitrary torsion free tensor $\bar{\nabla}_b$, we have the following holds by Eq. (3.1.7):
\begin{align*}
\nabla_a \omega_b = \bar{\nabla}_a \omega_b - C^c{}_{ab}\omega_c
\end{align*}
where $\omega_a$ is an arbitrary dual vector field, and $C_{ab}{}^c$ is a tensor. As $ {\nabla_b} f =\bar{\nabla}_b f $ by tensor property (4), we can set $\omega_b = {\nabla_b} f =\bar{\nabla}_b f $ and get:
\begin{align*}
\nabla_a \nabla_b f &= \bar{\nabla}_a \bar{\nabla} f_b - C^c{}_{ab}\nabla_c f \qquad\qquad\qquad 
\nabla_b \nabla_a f = \bar{\nabla}_b \bar{\nabla} f_a - C^c{}_{ba}\nabla_c f 
\end{align*}
Here $\bar{\nabla}$ is torsion free, that is we can write:
\begin{align*}
\nabla_a \nabla_b f - \nabla_b \nabla_a f = -C^c{}_{ab}\nabla_c f + C^c{}_{ba}\nabla_c f = (C^c{}_{ba}\nabla_c -C^c{}_{ab}\nabla_c)f
\end{align*}
Setting $-T^c{}_{ab} = C^c{}_{ba} -C^c{}_{ab}{}$ yields the result. 
\end{proof}
(b) Show that for any smooth vector fields $X^a$ and $Y^a$, we have the following holds:
\begin{align*}
T^{c}_{ab} X^aY^b = X^a\nabla_aY^c - Y^a \nabla_a X^c - [X,Y]^c
\end{align*}
\begin{proof}
\color{red}
Follows from direct computations:
\begin{align*}
[v,w]f = v^a \nabla_a (w^b \nabla_b f)- w^a \nabla_a (v^b \nabla_b f)= -T^c{}_{ab}v^aw^b \nabla_c f + (v^a \nabla_a w^c - w^a \nabla_a v^c) \nabla_c f
\end{align*}
from which is not hard to see that the result follows. 
\color{black}
\end{proof}

(c) Given a metric $g_{ab}$, show that there exists a unique derivative operator with torsion $T^c{}_{ab}$ such that $\nabla_cg_{ab} = 0$. Derive the analog of equation (3.1.29) expressing this derivative operator in terms of ordinary derivative and $T^c{}_{ab}$.\\

\begin{proof}
Let $\bar{\nabla}_a$ be any derivative operator. It suffices to show that there exists tensor $C_{ac}{}^d$ such that we have:
\begin{align*}
0 = \nabla_a g_{bc} = \bar{\nabla}_a g_{bc} - C^d{}_{ab}g_{dc} - C^d{}_{ac}g_{bd} = \bar{\nabla}_a g_{bc} - C_{cab} - C_{bac}
\end{align*}
Rearranging, we have:
\begin{align*}
\bar{\nabla}_a g_{bc} = C_{cab} + C_{bac}
\end{align*}
hence we can write:
\begin{align*}
\bar{\nabla}_a g_{bc} + \bar{\nabla}_bg_{ac} - \bar{\nabla}_cg_{ab} = C_{cab} + C_{bac} + C_{cba} + C_{abc} - C_{bca} - C_{acb} = T_{bac} + T_{abc} + 2C_{c(ab)}
\end{align*}
from which we see that the tensor $C$ is well-defined by rearrangement of the above equations, and making use of the fact $C_{abc} = C_{c(ab)} + C_{c[ab]} = C_{c(ab)} +T_{cab}/2$. Letting $\bar{\nabla}_a = \pd_a$, we obtain the following:
\begin{align*}
C_{abc} = \frac{1}{2}\left( \pd_a g_{bc} + \pd_b g_{ac} - \pd_c g_{ab} - T_{bac} + T_{abc} - T_{cab}\right)
\end{align*}
This gives the result.
\end{proof}


\newpage
\section{Chapter 3 Problem 4}
(a) Show that in $2$-dimensional manifold $M$, the Riemannian curvature tensor takes the form $R_{abcd} = Rg_{a[c}g_{d]b}$. 

\begin{proof}
By Chapter 3 Problem 3(b), we see that the Riemannian curvature tensor in $2$ dimension has $2^2(2^2-1)/12 = 1$ free component. Here we would like to show that $g_{a[c}g_{d]b}$ spans the vector space of tensors having the symmetries of the Riemannian tensor, we can write the following:
\begin{align*}
M_{abcd} =  g_{a[c}g_{d]b} = \frac{1}{2}\left( g_{ac}g_{db}-g_{ad}g_{cb}\right)
\end{align*}
by the symmetry of $g_{ab}$, we have $g_{ac} = g_{ca}$, hence we can write:
\begin{align*}
M_{abcd} = \frac{1}{2}\left( g_{ac}g_{db}-g_{ad}g_{cb}\right) 
&= -\frac{1}{2}\left( g_{bc}g_{da}-g_{bd}g_{ca}\right) = -M_{bacd}\\
&= -\frac{1}{2}\left( g_{ad}g_{cb}-g_{ac}g_{db}\right) = -M_{abdc}\\
&= \frac{1}{2}\left( g_{ca}g_{bd}-g_{cb}g_{ad}\right) = M_{cdab}
\end{align*}
Hence we have shown that tensor $M_{abcd}$ satisfies all symmetries required by $R_{abcd}$. It follows that we have $R_{abcd} = \mu M_{abcd}$ for some $\mu \in \R$ as both tensors have only $1$ free component, where we have $\mu = R$ follows from the definition of scalar curvature:
\begin{align*}
R =\mu g^{bd}g^{ac}M_{abcd} = \mu \left( (\text{tr}(g))^2 - (\text{tr}(g))\right)/2 = \mu
\end{align*}
This completes the proof. 
\end{proof}

(b) Show that in $3$-dimensions, the Weyl tensor vanishes identically. 
\begin{proof}
Here we have $R_{abcd} = C_{abcd}+M_{abcd}$ with $M_{abcd}$ as defined similarly in part (a). $C_{abcd}$ is traceless, hence contracting the second and fourth indices yields $R_{ac} = M_{ac}$ whose components satisfy the following: 
\begin{align}
R_{\alpha\gamma} =\sum_{\sigma} M_{\alpha\sigma\gamma}{}^\sigma 
\end{align}
here Eq. (1) is a system of $3(3+1)/2 = 6$ equations by the symmetry of $R_{ac}$. On the other hand, $R_{abcd}$ has $3^2(3^3-1)/12 = 6$ free components, which are completely determined by the system of equations described by Eq. (1), from which we conclude that the components of $C_{abcd}$ vanish identically. 
\end{proof}

\newpage
\section{Chapter 3 Problem 6}
The metric of Euclidean space $\R^3$ in spherical coordinates is defined by:
$$ds^2=  dr^2 + r^2(d\theta^2 + \sin^2(\theta) d\phi^2)$$
(a) Calculate the Christoffel components $\Gamma^\sigma{}_{\mu \nu}$ in spherical coordinate system. \\

Here we have:
\begin{align*}
\Gamma^\sigma{}_{\mu\nu} = \frac{1}{2}\sum_\sigma g^{\rho \sigma}\left( \frac{\pd g_{\nu \sigma}}{\pd x^\mu} + \frac{\pd g_{\mu \sigma}}{\pd x^\nu} - \frac{\pd g_{\mu\nu}}{\pd x^\sigma}\right)
\end{align*}
The nontrivial derivatives of the metric are the followings:
\begin{align*}
\frac{\pd g_{\theta\theta}}{\pd r} = 2r \qquad\qquad \frac{\pd g_{\phi\phi}}{\pd r} = 2r\sin^2(\theta) \qquad\qquad \frac{\pd g_{\phi\phi}}{\pd \theta} = 2r^2 \sin(\theta)\cos(\theta)
\end{align*}
Hence the only nontrivial components of the Christoffel symbol are the followings:
\begin{align*}
\Gamma^r{}_{\phi\phi}&=\frac{1}{2}\left( -g^{rr} \frac{\pd g_{\phi\phi}}{\pd r}\right) = - \frac{2r \sin^2(\theta)}{2}=-r\sin^2(\theta) \\
\Gamma^r{}_{\theta\theta}&= \frac{1}{2}\left(- g^{rr}\frac{\pd g_{\theta\theta}}{\pd r}\right) = -\frac{2r}{2} = -r  \\
\Gamma^\theta_{\phi\phi}&= \frac{1}{2}\left( -g^{\theta\theta} \frac{\pd g_{\phi\phi}}{\pd \theta}\right) =- \frac{1}{2}\frac{2r^2 \sin(\theta)\cos(\theta)}{r^2} = -\sin(\theta)\cos(\theta)\\
\Gamma^{\theta}{}_{r\theta} &= \frac{1}{2}\left(g^{\theta\theta} \frac{\pd g_{\theta\theta}}{\pd r} \right) =\frac{1}{2} \frac{2r}{r^2} = \frac{1}{r} \\
\Gamma^{\phi}{}_{\theta\phi} &= \frac{1}{2}\left(  g^{\phi\phi}\frac{\pd g_{\phi\phi}}{\pd \theta}\right) = \frac{1}{2}\frac{2r^2 \sin(\theta) \cos(\theta)}{r^2\sin^2(\theta)} = \cot(\theta)\\
\Gamma^{\phi}{}_{r\phi} &= \frac{1}{2}\left(  g^{\phi\phi}\frac{\pd g_{\phi\phi}}{\pd r}\right) = \frac{1}{2}\frac{2r \sin^2(\theta)}{r^2\sin^2(\theta)} = \frac{1}{r}
\end{align*}
(b) Write the components of the geodesic equation and verify the solution correspond to straight lines in Cartesian coordinates. \\

From Eq. (3.3.5), we have:
\begin{align*}
\frac{d^2 x^\mu}{dt}+ \sum_{\sigma,\mu} \Gamma^\nu{}_{\sigma\nu}\frac{d x^\sigma}{d t} \frac{d x^\nu}{d t}=0
\end{align*}
Utilizing part (a) we obtain the following system:
\begin{align*}
0&=\ddot{r} - r\dot{\theta}^2 - r\sin^2(\theta) \dot{\phi}^2 \\
0&=\ddot{\theta}+ {2\dot{r}\dot{\theta}}/{r} - \sin(\theta) \cos(\theta) \dot{\phi}^2 \\
0&= \ddot{\phi}+2\dot{r}\dot{\theta}/r + 2\cot(\theta) \dot{\theta}\dot{\phi}
\end{align*}
Setting $\dot{\theta} = \dot{\phi} = 0$ and letting $\dot{r}$ be an arbitrary constant is an obvious solution to the system, from which case we see that straight lines in Cartesian coordinates are solutions to the system. 

\newpage
\section{Chapter 4 Problem 1}
Show that Maxwell's Equations implies strict charge conservation $\nabla_aj^a = 0$.
\begin{proof}
From Maxwell's Equations, we can write:
\begin{align*}
\nabla^a F_{ab} = -4\pi j_b \qquad\Rightarrow \qquad \nabla^b\nabla^a F_{ab} = -4\pi\nabla^b j_b
\end{align*}
Here $F_{ab}$ is anti-symmetric, hence we have:
\begin{align}
\nabla^b\nabla^a F_{ab} -\nabla^a\nabla^b F_{ba} = -8\pi \nabla^b j_b
\end{align}
it suffices to show that the LHS of (2) vanishes. Here we can write:
\begin{align*}
\nabla^b\nabla^a F_{ab} -\nabla^a\nabla^b F_{ba} &= g^{ae}g^{bf}(\nabla_b\nabla_a - \nabla_a\nabla_b)F_{ef}\\
&= -g^{ae}g^{bf}(R_{abe}{}^dF_{df} + R_{abf}{}^dF_{ed})\\
&= -R^{fd}F_{df}-R^{ed} F_{ed}
\end{align*}
This completes the result as the Ricci curvature tensor is symmetric and $F_{ab}$ is anti-symmetric, that is $\nabla^b\nabla^a F_{ab} -\nabla^a\nabla^b F_{ba} = 0$.
\end{proof}

\newpage
\section{Chapter 4 Problem 8}
Two point masses of mass $M$ are attached to the ends of a spring of spring constant $K$. The spring is set into oscillation. In the quadrupole approximation, equation (4.4.58), what fraction of the energy of oscillation of the spring is radiated away during one cycle of oscillation?\\

Here let $L$ denote the distance between the masses, as $K$ is the spring constant, we have $L = L_0 + A \sin(\omega t)$ with $L_0$ being the equilibrium length of the spring, $\omega = \sqrt{2K/M}$, and $A$ being the amplitude of oscillation. From the quadrupole moment tensor, we have:
\begin{align*}
q_{\mu\nu} = 3 \sum_{i=1}^2 T^{00} x^{\mu}_i x^{\nu}_i 
\end{align*}
where $x_i^{\mu}$ is the position of the $i$-th mass. Here $T^{00} = \rho u_0u_0 + P(g_{00} + u_0u_0)$.



\hfill\break\hfill\break\hfill\break
\section{Chapter 4 Problem 9}
A binary star system consists of two star of mass $M$ and of negligible size in a nearly Newtonian circular orbit of radius $R$ around each other. Assuming the validity of equation (4.4.58), calculate the rate of increase of the orbital frequency due to emission of gravitational radiation.\\

\color{red}
Establish a relation between energy and frequency. Here we have: 
$$\text{Force}_{\text{centripetal}} = MR\omega^2 = M^2/(4R)$$
$$\text{KE} = MV^2 = M\omega^2 R^2 = M(M/(4R^3)) R^2 =M^2 /(4R) $$
\color{black}




\newpage
\section{Chapter 5 Problem 1}
The Robertson-Walker metric can be expressed in the following form:
\begin{align*}
ds^2 = -d\tau^2 + a^2(\tau)\left( \frac{dr^2}{1-kr^2} + r^2(d\theta^2 + \sin^2(\theta)\, d\phi^2\right)
\end{align*}
The case for flat space $k=0$ is trivial as one can perform change of variable $r = \psi$. Here we show the case for $k=1$, in which case we perform a change of variable $r = \sin(\psi)$ in Eq. (5.1.11), via tensor transformation, we can write the following:
\begin{align*}
(ds^2)_{rr} = \sum_{\mu, \nu} (ds^2)_{\mu\nu}\frac{\pd x^{\mu}}{\pd r}\frac{\pd x^{\nu}}{\pd r} = (ds^2)_{\psi\psi}\left( \frac{\pd \psi}{\pd r}\right)^2 = \frac{1}{\cos^2(\psi)} = \frac{1}{1-r^2}
\end{align*}
where we see the result follows as other components of $ds^2$ transform trivially. The transformation of Eq. (5.1.11) to the desired form for the case $k=-1$ can be obtained by letting $r = \sinh(\psi)$, in which case the similar calculation get:
\begin{align*}
(ds^2)_{rr} = \sum_{\mu, \nu} (ds^2)_{\mu\nu}\frac{\pd x^{\mu}}{\pd r}\frac{\pd x^{\nu}}{\pd r} = (ds^2)_{\psi\psi}\left( \frac{\pd \psi}{\pd r}\right)^2 = \frac{1}{\cosh^2(\psi)} = \frac{1}{1+r^2}
\end{align*}
This completes the result. \\

With $r$ ranging in $(-1,1)$, $\theta, \phi$ ranging in $(0,\pi)$, essentially half of the $3$-sphere is covered by the coordinate system $(r,\theta \phi)$. 
\newpage
\section{Chapter 5 Problem 2}
Here we derive Einstein's Equations (5.2.14) and (5.2.15) for the $3$-sphere and hyperboloid cases.\\

Choosing the coordinate system $(r,\theta,\phi)$ as in Chapter 5 Problem 1. Using Eq. (3.1.30), we can compute the Christoffel components, and applying Eq. (3.4.5), we obtain the nonvanishing Ricci components:
\begin{align*}
R_{\tau\tau} &= -{3\dot{a}'}/{a}\\
R_{rr} &= ({2k + 2\dot{a}^2 + \dot{a}\ddot{a}})/({1 - kr^2})\\
R_{\phi\phi} &= r^2 \sin^2(\theta)(2k+2\dot{a}^2 + a\ddot{a})\\
R_{\theta\theta} &= -1 + 2kr^2 - \cot^2(\theta) +\csc^2(\theta) + 2r^2 \dot{a}^2 + r^2 a\ddot{a}\\
\end{align*}
Using (3.2.32), we obtain:
\begin{align*}
R_{\tau\tau}+ \frac{1}{2}R = 8\pi \rho \qquad\qquad\qquad R_{**} - \frac{1}{2}R = 8\pi P
\end{align*}
where we have $R_{**} = R_{ab}u^au^b$ for $u^a$ being any unit vector tangent to the homogeneous hypersurfaces. Hence we have $R_{**} = (R_{ab}X^aX^b)/(g_{ab}X^aX^b) = R_{\phi\phi}/g_{\phi\phi}$ for the case $X^a = (\pd /\pd \phi)^a$. Combining the results yields the following expressions:
\begin{align*}
\frac{3k}{a^2} + \frac{3\dot{a}^2}{a^2} = 8\pi \rho \qquad\qquad\qquad
-\frac{k}{a^2} - \frac{\dot{a}^2}{a^2} - \frac{2\ddot{a}}{a} = 8\pi P
\end{align*}


\newpage
\section{Chapter 5 Problem 3}
Consider the modified Einstein's Equation with cosmological constant $\Lambda$. Write out the analogs of equations (5.2.14) and (5.2.15) with the $\Lambda$-terms. Show that static solutions of these equations are possible if and only if $k = 1$ and $\Lambda >0$. For a dust filled Einstein universe, relate the radius of the universe $a$ to the density $\rho$, and examine small perturbations from the equilibrium value of $a$ to show that the Einstein static universe is unstable.\\

Here we have:
\begin{align*}
G_{ab} + \Lambda g_{ab} = 8\pi T_{ab}
\end{align*}
and hence we have:
\begin{align*}
G_{\tau\tau} - \Lambda = 8\pi \rho\qquad\qquad\qquad G_{**}+\Lambda = 8\pi P
\end{align*}
As in Chapter 5 Problem 2, we see that we can write:
\begin{align}
\frac{3\dot{a}^2}{a^2}+ \frac{3k}{a^2} = 8\pi \rho + \Lambda\qquad\qquad\qquad
-\frac{k}{a^2} - \frac{\dot{a}^2}{a^2}-\frac{2\ddot{a}}{a} = 8\pi P - \Lambda\tag{*}
\end{align}
Suppose now static solution to (*) exists, that is $\dot{a} = 0$. From (*), eliminating $k$ we obtain:
\begin{align}
\frac{3\ddot{a}}{a} = -4\pi (\rho + 3P) +\Lambda\tag{**}
\end{align}
hence we see that (**) holds with $\ddot{a}=0$ only if $\Lambda>0$ as $\rho + 3P$ is positive. Then from (*), we deduce $k = 1$. Showing (**) admits a solution of $\dot{a} = 0$ is trivial. \\






\newpage
\section{Chapter 5 Problem 5}
Consider a radial null geodesic propagating in a Robertson-Walker cosmology. \\

(a) Show that for all three spatial geometries, the change in coordinate $\psi$ of the ray between times $\tau_1$ and $\tau_2$ is $\Delta \psi = \int_{\tau_1}^{\tau_2} d\tau/a$. \\

Suppose the geodesic is parametrized by $\lambda$, we can write the following along the geodesic as it is null:
\begin{align*}
0 = -\left( \frac{d\tau}{d\lambda}\right)^2 + a^2\left(\frac{d\psi}{d\lambda}\right)^2 \qquad \Rightarrow \qquad \frac{d\tau}{d\lambda} = \pm a \frac{d\psi}{d\lambda}
\end{align*}
With $\lambda_1$ and $\lambda_2$ be the parameter corresponds to $\tau_1$ and $\tau_2$, we can compute:
\begin{align*}
\Delta \psi = \int_{\lambda_1}^{\lambda_2} \frac{d\psi}{d\lambda}\, d\lambda = \pm \int_{\lambda_1}^{\lambda_2} \frac{1}{a} \frac{d\tau}{d\lambda}\, d\lambda = \pm \int_{\tau_1}^{\tau_2} \frac{d\tau}{a}
\end{align*}
where the last equality in the equation resulted from the theorem of change of variable. This gives the result.\\

(b) Show that in the dust filled spherical model, a light ray emitted at the big bang travels precisely all the way around the universe by the time of the big crunch.\\

From Table 5.1, we have:
\begin{align*}
a = \frac{C}{2}(1-\cos(\eta)) \qquad\qquad\qquad \tau = \frac{C}{2}(\eta - \sin(\eta))
\end{align*}
Here from $\eta = 0$ to $\eta = 2\pi$, we have $a(\eta)$ going a full cycle such that $a(0) = a(2\pi) =0$. Here it is not hard to observe that we have:
\begin{align*}
\Delta \psi = \int_0^{2\pi} \frac{1}{a(\eta)}\frac{d\tau}{d\eta}  \,d\eta = \int_0^{2\pi} \frac{1-\cos(\eta)}{1-\cos(\eta)}\, d\eta = 2\pi
\end{align*}
As $\psi$ ranges from $0$ to $2\pi$, this gives the result. \\

(c) Show that in the radiation filled spherical model, a light ray emitted at the big bang travels precisely halfway around the universe by the time of the big crunch. \\

From Table 5.1, we have:
\begin{align*}
a = B \left( 1- (1- \tau/B)^2 \right)^{1/2}
\end{align*}
Here from $\tau = 0$ to $\tau = 2B$, we have $a(\tau)$ going a full cycle such that $a(0) = a(2B) =0$. Here it is not hard to observe that we have:
\begin{align*}
\Delta \psi  = \int_0^{2\pi} \frac{d\tau}{B \left( 1- (1- \tau/B)^2 \right)^{1/2}} = -\pi
\end{align*}
this completes the result. 
\end{document}


