\documentclass[11pt]{article}

\usepackage{xcolor}
\usepackage{mathtools}
\usepackage[legalpaper, margin=1in]{geometry}
\usepackage{amsmath}
\usepackage{amssymb}
\usepackage{paralist}
\usepackage{rsfso}
\usepackage{amsthm}
\usepackage[inline]{enumitem}   
\usepackage{wasysym}

\newtheoremstyle{break}
  {\topsep}{\topsep}%
  {\itshape}{}%
  {\bfseries}{}%
  {\newline}{}%
\theoremstyle{break}
\theoremstyle{break}
\newtheorem{axiom}{Axiom}
\newtheorem{thm}{Theorem}[section]
\newtheorem{lem}{Lemma}[thm]
\newtheorem{prop}[lem]{Proposition}
\newtheorem{corL}{Corollary}[lem]
\newtheorem{corT}[lem]{Corollary}
\newtheorem{defn}{Definition}[corL]

\newcommand{\R}{\mathbb{R}}
\newcommand{\N}{\mathbb{N}}
\newcommand{\Z}{\mathbb{Z}}
\newcommand{\Q}{\mathbb{Q}}
\newcommand{\A}{\mathcal{A}}
\newcommand{\J}{\mathcal{J}}
\newcommand{\T}{\mathcal{T}}
\newcommand{\Td}{\mathcal{T}_d}
\newcommand{\C}{\mathcal{C}}
\newcommand{\M}{\mathcal{M}}
\newcommand{\Complex}{\mathbb{C}}
\newcommand{\Power}{\mathcal{P}}
\newcommand{\ee}{\cdot 10}
\newcommand{\Intab}{[\,a,b\,]}
\newcommand{\spa}{\text{span}}

\newcommand{\note}{\color{red}Note: \color{black}}
\newcommand{\remark}{\color{blue}Remark: \color{black}}
\newcommand{\example}{\color{green}Example: \color{black}}
\newcommand{\exercise}{\color{green}Exercise: \color{black}}

\makeatletter
\def\@seccntformat#1{%
  \expandafter\ifx\csname c@#1\endcsname\c@section\else
  \csname the#1\endcsname\quad
  \fi}
\makeatother

\makeatletter
\newcommand*{\rom}[1]{\expandafter\@slowromancap\romannumeral #1@}
\makeatother

\makeatletter
% This command ignores the optional argument for itemize and enumerate lists
\newcommand{\inlineitem}[1][]{%
\ifnum\enit@type=\tw@
    {\descriptionlabel{#1}}
  \hspace{\labelsep}%
\else
  \ifnum\enit@type=\z@
       \refstepcounter{\@listctr}\fi
    \quad\@itemlabel\hspace{\labelsep}%
\fi}
\makeatother
\parindent=0pt


\begin{document}

\newpage
\example Consider the equality $y^5 + xy +z = 0$, call it $f(x,z,y) = 0$. \\Here we order the variables as $(x,z,y)$. Then we can write the following:
\begin{align*}
Df = \begin{bmatrix}
y & 1 & 5y^4+x
\end{bmatrix}
\end{align*}
We can use Implicit Function Theorem when we have $5y^4+x \neq 0$. We will identity the "folding curve" for the $f(x,z,y)=0$. Here we need $g(x,z,y)=0$ for $g(x,z,y) =\begin{bmatrix}
y^5 +xy+z \\ 5y^4+x
\end{bmatrix} $.\\
$$Dg(x,z,y) =\begin{bmatrix}
y & 1 & 5y^4+x \\ 1 & 0 & 20y^3
\end{bmatrix}  $$
Note here: $$\begin{bmatrix}
y & 1 \\ 1 & 0
\end{bmatrix}\text{ is always invertible}$$
We can choose $(x,z)$ being dependent variables and $y$ being independent variable.\\ Then $x = -5y^4,\ y=4z^5$.


From Wednesday handout:
\begin{align*}
f: \begin{bmatrix}
x \\y 
\end{bmatrix} \mapsto y^5 + xy +1 \\
Df = \begin{bmatrix}
y & 5y^4 + x
\end{bmatrix}\\
E \coloneqq f^{-1}(0) \text{ is locally } y = g(x) \text{except possibily when } 5y^4 + x = 0
\end{align*}

Assuming $y$ is differentiable with respect to $x$, then we have $$5y^4\frac{dy}{dx}+ y + x\frac{dy}{dx} = 0 \qquad\Rightarrow\qquad \frac{dy}{dx} = -\frac{y}{5y^4+x}$$
$$g'(x) = -\frac{g(x)}{5(g(x))^4 + x}$$


\begin{align*}
\text{let } \widetilde{g}(x) = \begin{bmatrix}
x \\ g(x) 
\end{bmatrix}\\
f\circ \widetilde{g} = 0 \text{ because we assume that the graph of g lies in E}\\
Df(\widetilde g(x))(\widetilde{g} ' (x)) = 0\\
\begin{bmatrix}
\frac{\partial f}{\partial x}\begin{bmatrix}
 x \\ g(x)
\end{bmatrix} & 
\frac{\partial f}{\partial y}\begin{bmatrix}
 x \\ g(x)
\end{bmatrix}\cdot \begin{bmatrix}
1 \\ g'(x)
\end{bmatrix} 
\end{bmatrix}= \frac{\partial f}{\partial x} \begin{bmatrix}
x \\ g(x)
\end{bmatrix} + \frac{\partial f}{\partial y}\begin{bmatrix}
x \\ g(x)
\end{bmatrix} \cdot g'(x) = 0\\
g'(x) = \frac{-\frac{\partial f}{\partial x} \begin{bmatrix}
x \\g(x)
\end{bmatrix}}{\frac{\partial f}{\partial y}\begin{bmatrix}
x \\ g(x)
\end{bmatrix}}
\end{align*}
In higher dimension, let $\widetilde{g}(\vec{x}) = \begin{bmatrix}
\vec{x} \\ g(\vec{x}) 
\end{bmatrix}$, $f\circ \widetilde{g}= 0$.
$(Df\circ \widetilde{g}) D\widetilde{g} = 0$
\begin{align*}
\Rightarrow \begin{bmatrix}
\frac{\partial f}{\partial \vec{x}}\circ \widetilde {g} & \frac{\partial f}{\partial y}\widetilde{g} 
\end{bmatrix} \begin{bmatrix}
 I \\ Dg
\end{bmatrix} = \frac{\partial f}{\partial \vec{x}}\circ \widetilde{g} + (\frac{\partial f}{\partial \vec{y}}\circ \widetilde{g}) Dg = 0\\
Dg(\vec{x}) = -\left(\left(\frac{\partial f}{\partial  \vec{y}} \right)^{-1} \frac{\partial f}{\partial \vec{x}} \right)(\widetilde{g}(\vec{x}))
\end{align*}






\end{document}
